\documentclass[bachelor,openright,notchinese]{sustcthesis}
% 默认twoside 双面打印
% 将master修改为bachelor, doctor or master
% 要使用adobe字体,添加adobefonts选项
% 使用euler数学字体,如不愿使用,去掉euler
% 使用外文写作,请添加notchinese

% 设置图形文件的搜索路径
\graphicspath{{figures/}}



%%%%%%%%%%%%%%%%%%%%%%%%%%%%%%
%% 封面部分
%%%%%%%%%%%%%%%%%%%%%%%%%%%%%%



 % 中文封面内容
  \title{六次循环域}%一般情况下扉页和封皮、书脊共用一个标题文本,可以不用定义\spinetitle(仅硕博有用), \covertitle(本硕博均有用)和\encovertitle(仅本科有用)。特殊情况见下。
  %特殊情况1:本例中\title命令里含有换行控制字符,这会导致制作书脊的时候出现错误,例如如果你注释掉\spinetitle{...}这一行就会报错。这时需要定义一个不含换行等命令的\spinetitle,这并不表示\spinetitle里不能有任何命令——只能使用有限的命令。
  %特殊情况2:本例中标题过长,所以需要缩小书脊标题的字号。
  %特殊情况3:本例中中英文混排,由于tex竖排的原理限制,中英文基线不重合,所以需要人工调整英文的基线。具体调整量根据不同字体有所不同。
  %\covertitle{六次循环域}
  %\covertitle{中文题目第一行\\中文题目第二行}
  %不要在此调整封皮字体大小! Do not set Cover Page font size here!
  %特殊情况4:本例中\title中含有多个换行,导致标题超过了两行。根据制本厂规定,封皮标题不能超过两行。因此需要定义封皮使用的标题\covertitle. 如果你注释掉这一行,就会发现封皮不符合规定。
  %\encovertitle{On Cyclic Sextic Fields}
  %\encovertitle{English Title Line 1\\English Title Line 2\\English Title Line 3}
  %不要在此调整封皮字体大小! Do not set Cover Page font size here!
  %特殊情况5:仅本科生有用。本科封皮中有英文标题,不超过三行。与上类似。
  \author{张 \ 文 \ 超}
  \depart{金融数学系}%系别,硕博请用系代号,本科请用全称如
  \major{数学专业}%专业,硕博请用全称,本科不需要
  \advisor{李景治\ 教授}
  \coadvisor{张贤科\ 教授,\ 余解台\ 教授}%第二导师,没有请注释掉
  \studentid{11110015}%For bachelor only
  \submitdate{二〇一四年九月}

  % 英文封面内容
  \entitle{On Cyclic Sextic Fields and Galois Extensions of Function Fields}
  \enauthor{Wenchao Zhang}
  \studentid{11110015}
  \endepart{Financial Mathematics}
  \enmajor{Mathematics}
  \enadvisor{Jingzhi Li}
  \encoadvisor{Xianke Zhang(Number Theory)}
  \encoadvisorsec{Jietai Yu(Galois Theory)}
  \ensubmitdate{September, 2014}
  
\begin{document}

% 封面
\maketitle

%特别注意,以下述顺序为准,在对应部分添加文档部件,切勿颠倒顺序:
%本科论文的文档部件顺序是:
%    frontmatter:致谢、目录、中文摘要、英文摘要、
%    mainmatter: 正文章节
%    backmatter: 参考文献或资料注释、附录
%%%%%%%%%%%%%%%%%%%%%%%%%%%%%%
%% 前言部分
%%%%%%%%%%%%%%%%%%%%%%%%%%%%%%
\frontmatter
\makeatletter

\ifustc@bachelor
	%%%%%%%%%%%%%%%%%
	%本科论文修改这里
	%%%%%%%%%%%%%%%%%
	% 致谢
	\chapter*{诚信承诺书}
\label{chap:honest}

\begin{enumerate}
\item 本人郑重承诺所呈交的毕业设计(论文),是在导师的指导下,独立进行研究工作所取得的成果,所有数据、图片资料均真实可靠。
\item 除文中已经注明引用的内容外,本论文不包含任何其他人或集体已经发表或撰写过的作品或成果。对本论文的研究作出重要贡献的个人和集体,均已在文中以明确的方式标明。
\item 本人承诺在毕业论文(设计)选题和研究内容过程中没有抄袭他人研究成果和伪造相关数据等行为。
\item 在毕业论文(设计)中对侵犯任何方面知识产权的行为,由本人承担相应的法律责任。
\end{enumerate}

\vskip 3\baselineskip


\begin{flushright}

作者签名: \underline{\hspace{4cm}}
\vskip \baselineskip
\underline{\hspace{1.4cm}}年\underline{\hspace{0.7cm}}月\underline{\hspace{0.7cm}}日

\end{flushright}



	\chapter{Preface}
\label{chap:chap-preface}
This thesis is intended to summarize the learning outcomes and some works done by me during the undergraduate period in south university of science and technology of china (SUSTC). In this thesis, I choose two topics, one is related to algebraic number theory and the other one is something about function fields. The central idea or tool is the Galois theory, which connects the two part for this paper. For the number fields part, I have rearrange many references and books in order to give a simple and clear skeleton. For the function fields, we prove an explicit theorem, which is useful for studying the discrete (finite) subgroups of the automorphism group of a function field.  

Before the main contents of the thesis, I would like to take some words for my process on studying mathematics during the time in SUSTC. Although there is no mathematics department for our newly established university, I am grateful that I have learn many mathematics, especially abstract algebra, algebraic number theory and algebraic geometry from Prof. Xianke Zhang, and  Prof. Jietai Yu in the University of Hongkong etc.

As a student who loves mathematics very well, I have met so many kindly and responsible mathematicians during my 3.5-years' college life. During the first two years, as the 
representative of mathematics lessons in our inaugural class across all majors, I have ``obligation" to get good grades on mathematics lessons in order to set a good example. I learned Calculus and Analysis from Prof. Xuefeng Wang in mathematics department of Tulane University, Prof. Jinzhong Zhang in Guangzhou University, Prof. Zhongkan Liu in SUSTC and Beihang University, who is my first supervisor. We had frequent discussions on some topics on analysis as well as many other topics like elementary geometry and ordinary differential equation etc. 

After entering financial mathematics department, I also learned ordinary differential equation, probability, stochastic process, numerical methods, as well as some statistics from the professors in the department. Among of them, Prof. Anyue Chen, Prof. Jingzhi Li, Prof. Xuejun Jiang, Prof. Huaiqing Wang, Prof. Dejun Xie and Prof. Bianxia Sun gave me so many supports. Especially, Prof. Li, as my graduation designation supervisor, encouraged me a lot on studying algebra, and helped me to know some good mathematician friends of him.

Prof. Xianke Zhang, supervisor of me (after Prof. Liu) and an academic supervisor of this thesis, has guided me into the world of abstract algebra, a field which is quite abstruse for freshman. However, I found appealing thing inside the abstract definitions and complex relations one day. After that, besides the formal classes, we held several seminars on algebraic number theory and commutative algebra etc under the guidances of Prof. Zhang. 

Prof. Jietai Yu who has inspired me on Galois theory and algebraic geometry, has given me many guidances on research area as well as daily life. He gave us Galois courses as well as seminar on algebraic geometry. Among that time, I had my first opportunity to give a presentation in the formal seminar. I still remember the topics for that seminar is the finite subgroups of $\operatorname{PGL}(2,\mathbb{C})$ (we will show the relevant topic in chapter \ref{chap:chap-eight}). What's more, Prof. Yu found more opportunities for me to learn more mathematics as well as to further my study.

During those experiences outside the campus, I have learnt more recent works on affine algebraic geometry. The happiest things for me during those academic activities are finding many good friends, like Swapnil Lokhande, Sagar Kolte, Shameek Paul, Shihong Ma, Ju Huang, Haifeng Tian etc. They really gave me so much encouragement and concern. Also, I also met many good mathematician, like Prof. Alexey Belov, Prof. Leniod Makar-Limanov, Prof. Wenhua Zhao, Prof. Fang Li, Prof. Xiankun Du and so on. They gave me some instructions on some problems and encouraged me on studying mathematics.

Last but not least, I've come to realize that the ability of self-study is the most important good properties for a university student. I have paid much time to learn something outside the courses, to find some interesting topics, to sort out learning knowledge. During these years, I wrote a personal mathematics blog to record those things.


\vskip 28pt

\begin{flushright}

Wenchao ZHANG

September, 2014 at SUSTC

\end{flushright}






     
	
	
	%目录部分
	%目录
	\tableofcontents
	%默认表格、插图、算法索引名称分别为“表格索引”、“插图索引”和“算法索引”
	%如果需要自行修改lot,lof,loa的名称,请定义
	%\ustclotname{...}
	%\ustclofname{...}
	%\ustcloaname{...}

	% 表格索引
	%\ustclot
	% 插图索引
	%\ustclof
	%算法索引 
	%如果需要使用算法环境并列出算法索引,请加入补充宏包。
	%\ustcloa
	
	% 摘要
	\begin{cnabstract}
本文主要分为两个部分,前半部分主要研究了六次循环域的结构。首先将代数数论的一般理论、二次域和循环三次域的主要理论和近期文献中有关三次、六次循环域的结果, 进行了整理综合。 在此基础上, 我们给出了一般六次循环域的整基,素分解的方法,具体给出了多个例子。例如,我们解决了最简单的复六次循环域——7次分圆域的判别式,整基,素分解。同时,利用一般的类数和单位的理论,计算了它的类数和单位群。其次,我们给出了一个实六次循环域,在解决该例子时先利用线性预解子的方法判定出所给多项式$f(x)=x^6-x^5-6x^4+6x^3+8x^2-8x+1$的伽罗瓦群为$C_6$,即得到其对应的分裂域为六次循环域。之后,通过六次循环域的结果给出该例子的二次子域和三次循环子域,从而进一步得到它的整基、素分解,并且我们还计算了这个域的类数。

六次循环域整基和素分解的一般结果是在利用解析数论方法得到判别式的前提下,利用其子域结构得出来的。

第二部分研究的是函数域上的伽罗瓦扩张。在经典的代数和代数几何理论中,L\"{u}roth定理揭示了一元函数域$K(x)$和基域$K$的中间域$E$是基域的单扩张(中间域不等于基域时为非代数扩张)。进一步地,本文假定$K(x)/E$为伽罗瓦扩张,不利用L\"{u}roth定理,证明了$E=K(u)$,其中$u\in K(x)$,可以被$\operatorname{Gal}(K(x)/E)$的初等对称多项式所确定。我们在其后也给出了一个经典$\operatorname{Gal}(K(x)/E)=D_3$的例子。

\keywords{伽罗瓦理论,六次循环域,整基,素分解,L\"{u}roth定理}
\end{cnabstract}

\begin{enabstract}
We have two parts in this thesis. In the first part, we study the structure of cyclic sextic field with many details on its subfields: quadratic field and cyclic cubic field. We reorganize main theory of algebraic number theory and some recent references on cyclic cubic field and cyclic sextic field. Based on these results, we solve the integral basis of the cyclic sextic field as well as find the prime decomposition algorithm. Precisely, we give some examples applying our results. For example, we solve discriminant, integral basis, prime decomposition of 7-cyclotomic field. We also compute its unit group and class number using general theory. We also give an example for real cyclic sextic field. Using linear resolvent method, we first determine that the minimal polynomial has Galois group $C_6$. Then we also find the quadratic subfield and cyclic cubic subfield in order to get the integral basis and prime decomposition. We compute the class number as well.

The general results for integral basis and prime decomposition of cyclic sextic field are given by structures of its subfields based on its discriminant which is computed through analytic number theory's method. 

In the second part, we discuss Galois extensions of a function field. In the classical theory of algebra and algebraic geometry, L\"{u}roth's theorem reveals that any intermediate field $E/K$ of $K(x)/K$ (where $x$ is transcendental extension over $K$) is a simple extension. More precisely, assuming that $K(x)/E$ is Galois, without using L\"{u}roth's theorem, we prove that $E=K(u)$, where $u\in K(x)$ can be determined by the elementary symmetric polynomials. We then give a classical example for $\operatorname{Gal}(K(x)/E)=D_3$.   

\enkeywords{Galois Theory,Cyclic Sextic Fields, Integral Basis, Prime Decomposition, L\"{u}roth's Theorem}
\end{enabstract}
%此文件中含有中英文摘要
    \begin{denotation}
\item[$\mathbb{Q}$] rational number field
\item[$\mathbb{Z}$] rational integer ring
\item[$K$] algebraic number field
\item[$O_K$] the ring of integers of $K$
\item[$\operatorname{Aut}(K/F)$] Automorphism group of $K$ which fixes $F$
\item[$\operatorname{Gal}(K/F)$] Galois group of $K$ which fixes $F$
\item[$\operatorname{Res}(f,g)$] resultant of polynomials $f(x)$ and $g(x)$
\item[$R_{p,f}$] resolvent polynomial of $f(x)$ with $orb(p)$
\item[$\operatorname{Disc}(f)$] discriminant of polynomial $f$
\item[$\operatorname{Disc}(\theta)$] discriminant of minimal polynomial of $\theta$,i.e. $\operatorname{Disc}(1,\theta,\dots,\theta^{n-1})$
\item[$\operatorname{ht}(f)$] height (or degree) of a rational function $f$
\item[$d(K)$] discriminant of field $K$
\item[$\operatorname{N}(\alpha)$] Norm of $\alpha$
\item[$\operatorname{Tr}(\alpha)$] Trace of $\alpha$
\item[$C_n$] cyclic group
\item[$Cl(K)$] class group of $K$
\item[$h(K)$] class number of $K$
\item[$U(K)$] group of units in $K$
\item[$\zeta_n$] $n^{th}$ root of unity

\end{denotation}

    
\else
	%%%%%%%%%%%%%%%%%
	%硕博论文修改这里
	%%%%%%%%%%%%%%%%%
	% 摘要
	\begin{cnabstract}
本文主要分为两个部分,前半部分主要研究了六次循环域的结构。首先将代数数论的一般理论、二次域和循环三次域的主要理论和近期文献中有关三次、六次循环域的结果, 进行了整理综合。 在此基础上, 我们给出了一般六次循环域的整基,素分解的方法,具体给出了多个例子。例如,我们解决了最简单的复六次循环域——7次分圆域的判别式,整基,素分解。同时,利用一般的类数和单位的理论,计算了它的类数和单位群。其次,我们给出了一个实六次循环域,在解决该例子时先利用线性预解子的方法判定出所给多项式$f(x)=x^6-x^5-6x^4+6x^3+8x^2-8x+1$的伽罗瓦群为$C_6$,即得到其对应的分裂域为六次循环域。之后,通过六次循环域的结果给出该例子的二次子域和三次循环子域,从而进一步得到它的整基、素分解,并且我们还计算了这个域的类数。

六次循环域整基和素分解的一般结果是在利用解析数论方法得到判别式的前提下,利用其子域结构得出来的。

第二部分研究的是函数域上的伽罗瓦扩张。在经典的代数和代数几何理论中,L\"{u}roth定理揭示了一元函数域$K(x)$和基域$K$的中间域$E$是基域的单扩张(中间域不等于基域时为非代数扩张)。进一步地,本文假定$K(x)/E$为伽罗瓦扩张,不利用L\"{u}roth定理,证明了$E=K(u)$,其中$u\in K(x)$,可以被$\operatorname{Gal}(K(x)/E)$的初等对称多项式所确定。我们在其后也给出了一个经典$\operatorname{Gal}(K(x)/E)=D_3$的例子。

\keywords{伽罗瓦理论,六次循环域,整基,素分解,L\"{u}roth定理}
\end{cnabstract}

\begin{enabstract}
We have two parts in this thesis. In the first part, we study the structure of cyclic sextic field with many details on its subfields: quadratic field and cyclic cubic field. We reorganize main theory of algebraic number theory and some recent references on cyclic cubic field and cyclic sextic field. Based on these results, we solve the integral basis of the cyclic sextic field as well as find the prime decomposition algorithm. Precisely, we give some examples applying our results. For example, we solve discriminant, integral basis, prime decomposition of 7-cyclotomic field. We also compute its unit group and class number using general theory. We also give an example for real cyclic sextic field. Using linear resolvent method, we first determine that the minimal polynomial has Galois group $C_6$. Then we also find the quadratic subfield and cyclic cubic subfield in order to get the integral basis and prime decomposition. We compute the class number as well.

The general results for integral basis and prime decomposition of cyclic sextic field are given by structures of its subfields based on its discriminant which is computed through analytic number theory's method. 

In the second part, we discuss Galois extensions of a function field. In the classical theory of algebra and algebraic geometry, L\"{u}roth's theorem reveals that any intermediate field $E/K$ of $K(x)/K$ (where $x$ is transcendental extension over $K$) is a simple extension. More precisely, assuming that $K(x)/E$ is Galois, without using L\"{u}roth's theorem, we prove that $E=K(u)$, where $u\in K(x)$ can be determined by the elementary symmetric polynomials. We then give a classical example for $\operatorname{Gal}(K(x)/E)=D_3$.   

\enkeywords{Galois Theory,Cyclic Sextic Fields, Integral Basis, Prime Decomposition, L\"{u}roth's Theorem}
\end{enabstract}
%此文件中含有中英文摘要
	% 目录
	\tableofcontents
	%默认表格、插图、算法索引名称分别为“表格索引”、“插图索引”和“算法索引”
	%如果需要自行修改lot,lof,loa的名称,请定义
	%\ustclotname{...}
	%\ustclofname{...}
	%\ustcloaname{...}

	% 表格索引
	\ustclot
	% 插图索引
	\ustclof
	%算法索引 
	%如果需要使用算法环境并列出算法索引,请加入补充宏包。
	%\ustcloa
	
	%符号说明,需要加入补充包
	\begin{denotation}
\item[$\mathbb{Q}$] rational number field
\item[$\mathbb{Z}$] rational integer ring
\item[$K$] algebraic number field
\item[$O_K$] the ring of integers of $K$
\item[$\operatorname{Aut}(K/F)$] Automorphism group of $K$ which fixes $F$
\item[$\operatorname{Gal}(K/F)$] Galois group of $K$ which fixes $F$
\item[$\operatorname{Res}(f,g)$] resultant of polynomials $f(x)$ and $g(x)$
\item[$R_{p,f}$] resolvent polynomial of $f(x)$ with $orb(p)$
\item[$\operatorname{Disc}(f)$] discriminant of polynomial $f$
\item[$\operatorname{Disc}(\theta)$] discriminant of minimal polynomial of $\theta$,i.e. $\operatorname{Disc}(1,\theta,\dots,\theta^{n-1})$
\item[$\operatorname{ht}(f)$] height (or degree) of a rational function $f$
\item[$d(K)$] discriminant of field $K$
\item[$\operatorname{N}(\alpha)$] Norm of $\alpha$
\item[$\operatorname{Tr}(\alpha)$] Trace of $\alpha$
\item[$C_n$] cyclic group
\item[$Cl(K)$] class group of $K$
\item[$h(K)$] class number of $K$
\item[$U(K)$] group of units in $K$
\item[$\zeta_n$] $n^{th}$ root of unity

\end{denotation}
%不是必需的,如果不想列出请注释掉
\fi
\makeatother

%%%%%%%%%%%%%%%%%%%%%%%%%%%%%%
%% 正文部分
%%%%%%%%%%%%%%%%%%%%%%%%%%%%%%
\mainmatter
  \chapter{Introduction}
\label{chp:introduction}
Formally, there are 2 parts and 8 chapters in this thesis. The first 4 chapters, namely chapter 2-4, mainly refer to some reviews for Galois theory and Algebraic Number theory etc. In chapter 6-8, we obtain some results on cyclic sextic fields and Galois extensions of function fields. And then, we also give some examples using our consequences. 

Algebraic Number Theory and Algebraic Geometry are two central researching frontiers of pure mathematics. To find the discriminant, integral basis, class number of a algebraic number field is a classical question in algebraic number theory. More accurately, if we consider the splitting field of the polynomials over $\mathbb{Q}$, which is certainly Galois, then some restrictions from Galois extension will affect the structures of the field. Igor Shafarevich showed that every finite solvable group $G$ is realizable over $\mathbb{Q}$, i.e. there exists a field $K$ such that $\operatorname{Gal}(K/\mathbb{Q})\cong G$. This is related to a famous question, the Inverse Galois Problem. For our cases, we focus on some simple cyclic fields which have cyclic group as its Galois group over $\mathbb{Q}$. 

We introduce Galois theory in chapter \ref{chap:chap-three}. The Galois theory is the principal connection between the two parts of the thesis. In chapter \ref{chap:chap-three}, we will first introduce some necessary preliminaries of Galois Theory for our results based on Emil Artin's book \citep{artin1944galois} and David A. Cox's book\citep{cox2012galois}. Then, we focus on computing the Galois group of a polynomial. We sort out some useful materials from A. Healy \citep{healy2002resultants}, C. Bright \citep{bright2013computing}, and L. Soicher and J. McKay \citep{soicher1985computing}.

One will find that chapter 3-5 have almost the same content's structure, i.e. we have several common sections like the discriminant, integral basis, decomposition of prime, unit group and class number and so on. In fact, Chapter 3 is the reviews of general theory for number fields and some analytic number theory. We have paid much time on sorting out the materials from books of Prof. Xianke Zhang \citep{Xianke2006ANT}, Henri Cohen \citep{cohen1993course} and Richard Molin \citep{mollin1999algebraic}. In chapter 4 and 5, we have clearly rearranged those results for quadratic field and cyclic cubic field based on Cohen and Zhang's books. Some of results has been modified for the simplicity. In chapter 5, we have posed an example for prime decomposition in cyclic cubic field and two examples for computing the class number of those fields. What's more, we have recovered the results of Seidelmann's paper \citep{seidelmann1917gesamtheit} through a lemma in Cohen's book. We also give a formula (See theorem \ref{thm:ccuformula}) for a type of cyclic cubic field which is given risen by the ideas of F.C. Orvay's \citep{orvay1991cyclic}.

In chapter 6, we first refer to Sirpa M{\"a}ki's results \citep{maki1980determination} on discriminant and conductor of a cyclic sextic field, and then get the integral basis and prime decomposition of it.  M{\"a}ki give a direct result for discriminant, we have tried to recover the result to get it using the conductor-discriminant formula. Then with this message, we complete to find the integral basis of cyclic sextic field. At section \ref{sec:primdsex}, we succeed in obtaining all the cases for decomposition of prime numbers. For real cyclic cubic field, the unit group and class number has been solved for some ``small" (actually is quite large) discriminants, we can find a table in M{\"a}ki's book. For complex cyclic sextic field, it's a CM-field, hence the class number and the unit group could be reduced into its cyclic cubic field. But unfortunately, we haven't got the final precise results of them.

In chapter 7, we give two examples. One of them is a complex cyclic sextic field and the other is a real cyclic sextic field. For this a complex cyclic sextic field, i.e. 7-th cyclotomic field, we use our results in chapter 6 to get its discriminant and the prime decomposition, integral basis. Then we verify those results through the theory of  
cyclotomic field. Since this field is quite simple, we use the general theory of class number to find its class number is 1. Then, we also given an inconspicuous example. We first verify the Galois group of the polynomial using the method in chapter 2, then we first compute the polynomial discriminant. After that, using the discriminant formula, we confirm the subfields of it, and also do other processes as we mentioned above except the unit group. We also find another example proposed by A Bremner and B Spearman\citep{bremner2010cyclic} with sextic trinomial, but unfortunately, we haven't got the final results.

Although there is only one chapter for part two, i.e. the Galois extensions in a function field, this is still an important part of my thesis. We have proved two theorem for specializing L\"{u}roth's theorem within Galois extension. More precisely, assuming that $K(x)/E$ is Galois, without using L\"{u}roth's theorem, we prove that $E=F(u)$, where $u\in K(x)$ can be determined by the elementary symmetric polynomials. We then give a classical example for $\operatorname{Gal}(K(x)/E)=D_3$. 


	
  %自行添加
  \chapter{Galois Theory}
\label{chap:chap-three}
Galois Theory, named after \'{E}variste Galois, is a useful tool to provide a connection between field theory and group theory. From the fundamental theorem of the Galois theory, we could find that the certain problems in field theory can be reduced to group theory, which is in some sense simpler and better understand. Originally, Galois used permutation groups to describe how the various roots of a given polynomial equation are related to each other. The modern approach to Galois theory, developed by Richard Dedekind, Leopold Kronecker and Emil Artin, among others, involves studying automorphisms of field extensions. In this chapter, we will introduce some basic theory of Galois theory and an application of it.


\section{Galois Extension}
For simplicity, we skip the theory of group and some basic definition of field. A field $K$ containing a field $F$ is called an extension field of $F$. Such an $K$ can be regarded as an $F$-vector space, and we write $[K:F]$ for the dimension, which we called degree of the extension. 

Consider fields $K/F$, we say the extension $K/F$ is \textbf{algebraic} if for any $\alpha\in K$ is algebraic over $K$, i.e.  every element of $K$ is a root of some non-zero polynomial with coefficients in $F$. The field extensions that are not algebraic are called transcendental.  

we say an algebraic field extension $K/F$ is \textbf{separable} if for every $\alpha\in K$, the minimal polynomial of $\alpha$ over F is a separable polynomial, i.e., has distinct roots. For $F$ is characteristic $0$, any algebraic extension is separable. Also, for any finite field, any algebraic extension of it is separable. 

A \textbf{splitting field} of a polynomial $p(X)$ over a field $F$ is a field extension $K/F$ which $p$ factors into linear factors and such that the roots generate $K$ over $F$. We say an algebraic field extension $K/F$ is \textbf{normal} if $K$ is the splitting field of a family of polynomials in $F[x]$. Or equivalently, every irreducible polynomial in $F[X]$ that has one root in $K$, has all of its roots in $K$.

An $F$-isomorphism $K\rightarrow K$ is called an $F$-\textbf{automorphism} of $K$. The $F$-automorphisms of $K$ form a group, which we denote $\operatorname{Aut}(K/F)$.

The normal extension $K/F$ is also equivalent to that every embedding (i.e. injective ring homomorphism) $\sigma$ of $K$ into $F^c$ is an automorphism of $K$ over $K$.

\begin{definition}[Galois Extension]
A finite extension $K$ of $F$ is said to be Galois if $K/F$ is both separable and normal. 
The $F$-automorphisms of $K$ is called the \textbf{Galois group} of $K$ over $F$, and it is denoted by $\operatorname{Gal}(E/F)$.
\end{definition}

An important theorem of Emil Artin \citep{artin1944galois} states that for a finite extension $K/F$, each of the following statements is equivalent to the statement that $E/F$ is Galois: 
\begin{theorem}[Artin]
For an extension $K/F$, the following statements are equivalent:
\begin{enumerate}
\item $K$ is the splitting field of a separable polynomial $f\in F[x]$.
\item $F=K^G$ for some finite group $G$ of automorphisms of $K$.
\item $K$ is normal, separable and of finite degree over $F$.
\item $K/F$ is Galois.
\end{enumerate}
\end{theorem}

\section{The fundamental theorem of Galois theory}
We then call the fundamental theorem of Galois theory which is the central theorem of Galois theory.
\begin{theorem}[The fundamental theorem of Galois theory]
Let $K/F$ is Galois, and $G=\operatorname{Gal}(K/F)$. The maps $H\mapsto K^H$ and $M\mapsto \operatorname{Gal}(K/M)$ are inverse bijections between the set of subgroups of $G$ and the set of intermediate fields between $K$ and $F$. Moreover, we have 
\begin{enumerate}
\item the correspondence is inclusion-reversing.
\item indexes equal degrees: $(H_1:H_2)=[K^{H_2}:K^{H_1}]$  
\item $\sigma H\sigma^{-1}\leftrightarrow\sigma M$
\item $H$ is normal subgroup of $G$ if, and and only if, $K^H$ is normal over $F$, in which case $$\operatorname{Gal}(K^H/F)\cong G/H.$$
\end{enumerate}
\end{theorem}

\section{Galois Groups of Polynomials}
If the polynomial $f\in F[x]$ is separable, then its splitting field $F_f$ is Galois over $F$, and we call $\operatorname{Gal}(F_f/F)$ the Galois group $G_f$ of $f$.
From now on, we just consider the simplest case, i.e. $F=\mathbb{Q}$, and we denote $\operatorname{Gal}(f)=\operatorname{Gal}(\mathbb{Q}_f/\mathbb{Q})$. Then, any splitting field $\mathbb{Q}_f$ is Galois over $\mathbb{Q}$.

A well-known theorem is that the roots of $f$ are solvable in radical if only if $\operatorname{Gal}(f)$ is solvable, which provides some motivation as to why the Galois group of a polynomial is of interest.

Let $f$ be a univariate polynomial with rational coefficients. Throughout this article we suppose that $f$ has degree $n$ and roots $r_1,r_2,\dots,r_n$. The splitting field $\mathbb{Q}_f$ of $f$, denoted by $\mathbb{Q}(r_1,\dots,r_n)$, is a finite extension of $\mathbb{Q}$ generated by the roots of $f$. Then the Galois group of $f$ is
defined to be \[G_f=\operatorname{Gal}(f):=\operatorname{Gal}(\mathbb{Q}(r_1,\dots,r_n)/\mathbb{Q}).\] 

Observing that the splitting field $\mathbb{Q}_f$ of a monic polynomial $f(x)$ with rational coefficients can be change into a splitting field of a monic polynomial with integer coefficients. 

Now we use some words to explain it. Let $$f(x)=x^n+a_{n-1}x^{n-1}+\cdots+a_1 x+a_0,$$ where $a_i\in\mathbb{Q},i=0,\dots,n-1$. Assume the greatest common divisor of the denominators is $d$, then $a_i=b_i/d,$ where $b_i,d\in\mathbb{Z}, i=0,1,\dots,n-1$. Hence, we have 
$$f\left(\frac{x}{d}\right)=\left(\frac{x}{d}\right)^n+\frac{b_{n-1}}{d}\left(\frac{x}{d}\right)^{n-1}+\cdots+\frac{b_0}{d},$$
Whence, $$d^nf\left(\frac{x}{d}\right)=x^n+b_{n-1}x^{n-1}+\cdots+b_0,$$
It's clear that the splitting fields of $f(x)$ and $g(x):=d^nf(\frac{x}{d})$ coincide. More precisely, we called $f(x)$ is equivalent to $g(x)$ (resp. $G_f$ is equivalent to $G_g$) \textbf{up to scaling}.

If $f(x)\in K[x]$ is a separable irreducible polynomial of degree $n$ and $G_f$ is its Galois group over $K$, then the group $G_f$ can be embedded into $S_n$ by writing the roots of $f(x)$ as $r_1,\dots,r_n$ and identifying each automorphism in the Galois group with the permutation it makes on the $r_i$'s. 

In $S_n$ we have the alternating group $A_n\subset S_n$, the following result is famous to determine the Galois group of the polynomial:
\begin{theorem}\label{thm:discriminantsq}
Let $K=\mathbb{Q}(\theta)$ be a number field of degree $n$, where $\theta$ is an algebraic integer with $f(x)$ be a minimal polynomial, then $G_{f}\subset A_n$ if and only if $\operatorname{Disc}(f)$ is a square. 
\end{theorem}

\section{Dedekind's Criterion}
A theorem of Dedekind \citep{Conrad_recognizinggalois} which provides useful information about $G_f$ over $\mathbb{Q}$. 
\par

\begin{theorem}[Dedekind's criterion]\label{thm:Dedekindcri}
Let $f(x)\in\mathbb{Z}[x]$ be a monic irreducible polynomial of degree $n$. Put $f_p(x)=f(x) \mod p$. Suppose $f_p(x)$ is a product of monic irreducible polynomials of degrees $n_1,n_2,\dots,n_r$ in $\mathbb{F}_p[x]$, where $n_1+n_2+\cdots+n_r=n$. Then $G_f$ is a subgroup of $S_n$ which contains a permutation permuting the roots with cycles type $(n_1,n_2,\dots,n_r)$.
\end{theorem}
\par

From the theorem, we should know first that $f(x)$ is irreducible over $\mathbb{Q}$. Irreducibility can be read off from the factorizations, since a factorization over $\mathbb{Q}$ can be scaled to be  a (monic) factorization over $\mathbb{Z}$. then if we have linear factor in $\mathbb{Z}[x]$, then of course, we have linear factor in $\mathbb{Z}_p[x]$.

It is important to remember that Dedekind's theorem does not correlate any information about the permutations coming from different primes. We don't know the exact roots which permuted by the cycles.

%Now we give an example of this theorem:

%Let $f(x)=x^6+x^4+x+3$. Here are the factorizations of $f(x)$ modulo the first few primes:
%\begin{eqnarray*}
%&&f(x)\equiv (x+1)(x^2+x+1)(x^3+x+1) (\operatorname{mod} 2)\\
%&&f(x)\equiv x(x+2)(x^4+x^3+2x^2+2x+2) (\operatorname{mod} 3)\\
%&&f(x)\equiv (x+3)^2(x^4+4x^3+3x^2+x+2) (\operatorname{mod} 5)\\
%&&f(x)\equiv (x^2+5x+2)(x^4+2x^3+3x^2+2x+5) (\operatorname{mod} 7)\\
%&&f(x)\equiv (x+6)(x^5+5x^4+4x^3+8x^2+x+6) (\operatorname{mod} 11)\\
%&&f(x)\equiv (x^2+8x+1)(x^2+9x+10)(x^2+9x+12) (\operatorname{mod} 13)
%\end{eqnarray*}
%\par
%From the factorization modulo 7 or 13, we see there is no linear factor as well as no cubic factor, form $p=11$, there is no quadratic factor. Hence, $f(x)$ is irreducible. Moreover,from the factorizations modulo 2 and 3, we have $G_f$ contains permutations of cycle type (1,2,3) and (1,1,4) (namely 4-cycle). The factorization mod 5 does not tell us anything by Dedekind's criterion, because there is a multiple factor. From the later primes, we see $G_f$ contains permutations of the roots with cycle types $(2,4)$, $(1,5)$ (a 5-cycle), and $(2,2,2)$



\section{Computing Galois Group of a Polynomial}
Galois groups are not easy to compute. As Galois says in the ``Discours Preliminaire" to his first memoir on Galois theory \citep{cox2012galois}: 
\begin{quote}
If now you give me an equation that you have chosen at will, and about which you want to know if it is or is not solvable by radicals, I cannot do any more than indicate the means for answering your question, without wanting to charge either myself or any other person with doing it. In a word, the calculations are impractical. 
\end{quote}
Even with the aid of modern computers, it is not easy to compute the Galois group of a polynomial of large degree unless the polynomial has 
some special structure. We will introduce some ways of computing Galois groups of arbitrary polynomials.

Throughout the method of Dedekind's criterion, one can only determine whether the Galois group of a polynomial with degree $n$ in $\mathbb{Q}[x]$ is to be $S_n$ or $A_n$ or not. For more results on that, one can refer \citep{Conrad_recognizinggalois} etc.

The following Resolvent Method has developed by L. Soicher and J. Mckay etc. can be useful to compute the Galois group of the polynomial. This method was first described by Jordan\citep{jordan1870traite} in 1870, and improved by 
L. Soicher etc.\citep{soicher1985computing} who raised the linear resolvent polynomial method to computing the Galois group of a polynomial. We note two immediate consequences. First, $f$ is a separable polynomial, i.e., it has distinct roots. Second, $\operatorname{Gal}(f)$ is a transitive group, i.e. for all $r_i$ and $r_j$ there is some $\sigma\in\operatorname{Gal}(f)$ which sends $r_i$ to $r_j$.

To using this method, we should first know some definitions. The \textbf{orbit} of a polynomial $p\in R[x_1,\dots,x_n]$ under $S_n$ is the set of polynomials that $p$ can be sent to by permuting the $x_i$, and this is denoted by $\operatorname{orb}(p)$.

Note that, this definition can be thought of as measuring ``how close" a polynomial is to being symmetric. For example, orb{p} has the smallest situation, i.e. $\operatorname{orb}(p)=\{p\}$, then we have $p$ fixes any permutation, hence $p$ is symmetric polynomial.

The most important definition for the method is resolvent polynomial, here comes to our definition:
\begin{definition}
The \textbf{resolvent polynomial} is defined in terms of two polynomials $f\in\mathbb{Z}[x]$ and $p\in\mathbb{Z}[x_1,\dots,x_n]$ to be the new univariate polynomial
$$R_{p,f}(y):=\prod_{p_i\in\operatorname{orb}(p)}(y-p_i(r_1,\dots,r_n))$$
\end{definition}

In particular, a resolvent polynomial $R_{p,f}$, where $p=e_1x_1+\cdots+e_rx_r\in\mathbb{Z}[x]$ for some $r,1\leq r\leq n,$and $e_1,\cdots,e_r$ nonzero integers, is called a \textbf{linear resolvent polynomial}.

%Some examples are given following. Let $p=x_1+x_2$, then the resolvent polynomial of $f$ will have $\mathbf{C}_n^2$ roots. 
%Considering $f(x)=x^3-2$, then $\operatorname{orb}(p)=\{x_1+x_2,x_2+x_3,x_1+x_3\}$. It follows that 
%$$R_{p,f}(y):=(y-(r_1+r_2))(y-(r_1+r_3))(y-(r_2+r_3))=(y+r_3)(y+r_2)(y+r_1)=y^3+2$$
%Another two cases will be listed in the following table together with above polynomial:

%\begin{table}[htbp]
%\centering
%\begin{tabular}{c|c}
%\hline
%$f(x)$ & $R_{p,f}(y)$ \\
%\hline
%$x^3+2$ & $y^3+2$\\
%$x^4+1$ & $y^6-4y^2$\\
%$x^4+x^3+x^2+x+1$ & $y^6+3y^5+5y^4+5y^3-2y-1$\\
%\hline
%\end{tabular}
%\caption{Resolvent Polynomials of Given Polynomial with %$p=x_1+x_2$}
%\label{tab:resolvent}
%\end{table}

Since the resolvent is defined with respect to the orbit of $p$ it is symmetric in the $r_i$, i.e. permuting the $r_i$ will permute the roots of $R_{p,f}$ but does not change $R_{p,f}$ itself. i.e. the coefficients of $R_{p,f}$ are symmetric polynomials of $r_1,\dots,r_n$, and the by the \textbf{fundamental theorem of symmetric polynomials} can be written in terms of the elementary symmetric polynomials in $r_1,\dots,r_n$. However, the elementary symmetric polynomials in $r_1,\dots,r_n$ are exactly the coefficients of $f$, therefore integers. Hence, $R_{p,f}\in\mathbb{Z}[y]$ providing $p\in\mathbb{Z}[x_1,\dots,x_n]$ and $f\in\mathbb{Z}[x]$.

For larger degree polynomials, it's hard to get the resolvent. However, computer becomes to our useful tool for computing coefficients of $R_{p,f}$ from above property. We can first approximate the roots of $f$ via numerical root-finding methods, form all combinations of the roots as specified by $p$, and then expand the product from the definition to find approximations of the coefficients of $R_{p,f}$. Since the coefficients are integers, if the approximations are known with sufficient accuracy then the approximations may simply be rounded to the nearest integer.

The action by $\sigma\in\operatorname{Gal}(f)$ on the roots of $R_{p,f}$ actually gives $Gal(R_{p,f})$. More precisely, let $\phi: \operatorname{Gal}(f)\rightarrow \operatorname{Gal}(R_{p,f})$ be defined so that $\phi(\sigma)$ is the action by $\sigma$ on the roots of $R_{p,f}$. Formally, a proposition which can be found in Cohen's book \citep{cohen1993course} is the following:
\begin{proposition}
If the roots of $R_{p,f}$ are distinct then $\operatorname{Gal}(f)=\phi(\operatorname{Gal}(R_{p,f}))$.
%\begin{proof}
%By the definition of $\phi$, we have $$\phi(\operatorname{Gal}(f))\subset\operatorname{Gal}(R_{p,f}),$$ i.e., if $\sigma$ is an automorphism then $phi(\sigma)$ is also an automorphism. For the other direction, since the roots of $R_{p,f}$ are built out of $f$'s roots, we have the splitting fields $\mathbb{Q}_{R_{p,f}}\subset\mathbb{Q}_{f}$ , from which, we have $\operatorname{Gal}(R_{p,f})\subset\operatorname{Gal}(f)$. By projecting down into $\mathbb{Q}_{R_{p,f}}$ we have $$\operatorname{Gal}(R_{p,f})\subset\phi(\operatorname{Gal}(f)).$$
%\end{proof}
\end{proposition}

As mentioned before, to use the above theorem we require that $R_{p,f}$ have distinct roots, however it will not always be the case. The \textbf{Tschirnhausen transformation} is an algorithm to get rid of this problem. For more information, one can refer to Cohen's book \citep{cohen1993course}.

%We have see from the examples in table \ref{tab:resolvent} that there are some "locally" transitive properties. For example, $f=x^4+x^3+x^2+x+1$,$p=x_1+x_2,$ we have $$R_{p,f}=(y^4+2y^3+4y^2+3y+1)(y^2+y-1).$$ This doesn't give us the Galois group of $R_{p,f}$ it does tell us its locally transitivity. i.e. the orbits of the action by $\text{Gal(f)}$ on the roots of $R_{p,f}$.In the above case the orbit-length partition is $(4,2)$, corresponding to the degrees of the irreducible factors of $R_{p,f}$.
%Suppose the $\operatorname{Gal}(f)=G$, now consider each permutation of $G$ on the roots of $R_{p,f}$, then we can get another orbit-length partition. Consider all transitive subgroup of $S_4$ for above example, we can get the following table:

%\begin{table}[htbp]
%\centering
%\begin{tabular}{ccccc}
%\hline
%$S_4$ & $A_4$ & $D_4$ & $V_4$ & $C_4$\\
%(6) & (6) & (4,2) &(2,2,2) & (4,2)\\
%\hline
%\end{tabular}
%\caption{Orbit-length Partition with $p=x_1+x_2$}
%\label{tab:orbit}
%\end{table}

%From this we conclude that in our previous example $\text{Gal(f)}$ must be either $D_4$ or $C_4$. To distinguish between these cases, we should try another new resolvent polynomial.

%Let $p=x_1-x_2$, then we get $$R_{p,f}=(y^4+5y^2+5)(y^4+5y+5)(y^4-4y+5),$$ so the orbit-length partition of the roots of $R_{p,f}$ under $\operatorname{Gal}(f)$ is $(4,4,4)$. As before, we can get the new table

%\begin{table}[htbp]
%\centering
%\begin{tabular}{ccccc}
%\hline
%$S_4$ & $A_4$ & $D_4$ & $V_4$ & $C_4$\\
%(12) & (12) & (8,4) &(4,4,4) & (4,4,4)\\
%\hline
%\end{tabular}
%\caption{Orbit-length Partition with $p=x_1-x_2$}
%\label{tab:orbit2}
%\end{table}

%Hence, we conclude that $\operatorname{Gal}(f)=C_4$.   

The linear resolvent method (together with recognizing squarefree of the discriminant of the polynomial, i.e. determined the Galois group is a subgroup of $A_n$ or not) can be applied for polynomial up to degree 7 \citep{soicher1985computing}. One can find a table from  L. Soicher and J. McKay's work \citep{soicher1985computing}. In additional, we repose some parts of this table in Appendix \ref{chap:appD}, see table \ref{tab:allorbits}.





%\section{The Kronecker-Weber Theorem}
%\subsection{Inverse Galois Problem}
%Let $K$ be an arbitrary filed and $G$ be a group. one says $G$ is realized over $K$ if there exists a field $L$ such that $Gal(L/K)\cong G$.
    
%In Galois Theory, the inverse Galois problem concerns whether or not every finite group is realized over rational number filed $\mathbb{Q}$. This problem is still open, However, some easy cases have been solved.
 
%Shafarevich citep{SchmidtSafaThem} showed that every finite solvable group is realizable over $\mathbb{Q}$.

%A simple example is that we can construct any finite cyclic group realized over $\mathbb{Q}$ through cyclotomic extension of $\mathbb{Q}$.




  \chapter{Algebraic Number Theory}
\label{chap:chap-two}
Algebraic number theory is a major branch of number theory that studies algebraic structures related to algebraic integers. Usually, it studies algebraic properties of the algebraic integers' ring such as factorization, the behaviour of ideals, and field extensions. In this chapter, we will introduce some definitions and useful results in algebraic number theory as well as some preliminaries of analytic number theory based on \citep{Xianke2006ANT,cohen1993course,mollin1999algebraic}. 

\section{Algebraic Numbers and Number Fields}
We first give the necessary background on algebraic numbers, number fields etc. Let $\alpha\in\mathbb{C}$. Then $\alpha$ is called an \textbf{algebraic number} if there exists $f(x)\in\mathbb{Z}[x]/{0}$ such that $f(\alpha)=0$. The number $\alpha$ is called an \textbf{algebraic integer} if, in addition, one can choose $f$ to be monic.(i.e. with leading coefficient equal to 1).

More generally, we can define the integral element of a ring(See \citep{Xianke2006ANT}) through similar definition.

%\begin{definition}
%Let $\alpha\in\mathbb{C}$ be an algebraic number, and $f(x)$ its minimal polynomial. The conjugates of $\alpha$ are all the $deg(f)$ roots of $f$ in $\mathbb{C}$.
%\end{definition}

A \textbf{number field} is a field containing $\mathbb{Q}$ which, considered
as a $\mathbb{Q}$-vector space, is finite dimensional. The number $d=[K:\mathbb{Q}]=\text{dim}_{\mathbb{Q}}K$ is called the degree of the number field $K$.

The \textbf{signature} of a number field is the pair $(r_1,r_2)$ where $r_1$ is the number of embeddings of $K$ whose image lie in $\mathbb{R}$, and $2r_2$ is the number of non-real complex embeddings, so that $r_1+2r_2=n$. If $T$ is an irreducible polynomial defining the number field $K$ by one of its roots, the signature of $K$ will also be called the signature of $T$.

The following proposition \citep{cohen1993course} shows that there are only two possibilities for the signature of a Galois extensions.
\begin{lemma}\label{lem:signaturegal}
Let $K$ be a Galois extension of $\mathbb{Q}$ of degree $n$. Then,
either $K$ is totally real $(r_1=n)$,or K is totally complex $(r_2=n/2)$ which can occur only if $n$ is even.
\end{lemma}

\section{Discriminants of Elements and Fields}
The definition of discriminants of polynomials can be found in the Appendix \ref{chap:appB} if need some reviews, now we will introduce the definition of discriminant of elements and fields.
Let $K$ be a number field of degree $n$, $\sigma_i$ be the $n$ embeddings of $K$ into $\mathbb{C}$, and $\alpha_j$ be the set of $n$ elements of $K$. Then we have $$\operatorname{Disc}(\alpha_1,\dots,\alpha_n)=\det(\sigma_i(\alpha_j))^2=\det(\operatorname{Tr}_{K/\mathbb{Q}}(\alpha_i\alpha_j))$$
In particular,If $K=\mathbb{Q}(\alpha)$, $f(x)$ is the minimal polynomial of $\alpha$, then $$\operatorname{Disc}(f)=\operatorname{Disc}(\alpha)=\operatorname{Disc}(1,\dots,\alpha^{n-1})$$

we denote by $O_K$ the ring of algebraic(rational) integers of $K$. Then we have that the ring $O_K$ is a free $\mathbb{Z}$-module of rank $n=\operatorname{deg}(K)$. Hence we can define the (absolutely) integral basis as follows:

\begin{definition}
A $\mathbb{Z}$-basis of the free module $O_K$ will be called an (absolutely)\textbf{integral basis} of K. The discriminant of an integral basis is independent of the choice of that basis, and is called the \textbf{discriminant of the field} $K$ and is denoted by $d(K)$.
\end{definition}

Similarly, we can define a \textbf{relatively integral basis}: Let $A$ be a commutative integral domain with 1, $K=\operatorname{Frac}(A)$,$L/K$ is an extension with $[L:K]=n$, $B$ is the integral closure of $A$ in $L$. If $B$ is a free $A$-module , i.e. $B=A \alpha_1\oplus\cdots\oplus A\alpha_n$. Then we call $\alpha_1,\dots,\alpha_n$ the $A$-basis of $B$(resp. integral basis of $L/K$.) Then we have $$\operatorname{Disc}(L/K)=(\operatorname{Disc}(\alpha_1,\dots,\alpha_n)),$$ which is the discriminant of $L/K$. 

Then, we will show a important theorem \citep{mollin1999algebraic} regarding the relationship of the discriminant of the minimal polynomial and the discriminant of the field. 

\begin{lemma}\label{lem:discpoly-field}
Let $T$ be a monic irreducible polynomial of degree $n$ in $\mathbb{Z}[x]$, $\theta$ a root of $T$, and $K=\mathbb{Q}(\theta)$. If $f=[O_K:\mathbb{Z}[\theta]]$, then $$\operatorname{Disc}(T)=d(K)f^2.$$
\end{lemma}

The number $f$ is called the \textbf{index} of $\theta$ in $O_K$. A theorem for recognizing the integral basis of a field is important. Moreover, we have general results for relative integral basis \citep{Xianke2006ANT}. The following proposition has combined these two cases.

\begin{proposition}
The algebraic numbers $\alpha_1,\dots,\alpha_n$ form an integral basis if and only if they are algebraic integers and if $\operatorname{Disc}(\alpha_1,\dots,\alpha_n)=d(K)$.
More generally, if $L/K$ has integral basis, then $\beta_1,\dots,\beta_n\in B$ form an integral basis if and only if $(\operatorname{Disc}(\beta_1,\cdots,\beta_n))=\operatorname{Disc}(L/K)$.
\end{proposition}

The result related the structure of discriminant of a field due to Stickelberger can not be avoid when we talking about discriminant here:
\begin{lemma}[Stickelberger's criterion]\label{thm:stickelberger}
Let $K$ be a number field, then $$d(K)\equiv 0\text{ or }1(\operatorname{mod} 4)$$
\end{lemma}
 
The determination of an explicit integral basis and of the discriminant of a number field is not an easy problem, and is one of the main tasks of this article. However one case in which the result is trivial:
\begin{corollary}
Let $f$ be a monic irreducible(i.e. minimal) polynomial in $\mathbb{Z}[x]$, $\theta$ a root of $f$, and $K=\mathbb{Q}(\theta)$. Assume that the discriminant of $f$ is squarefree or is equal to $4d$ where $d$ is squarefree and not congruent to 1 modulo 4. Then the discriminant of K is equal to the discriminant of $f$, and an integral basis of $K$ is given by $1,\theta,\dots,\theta^{n-1}$.
\end{corollary}

Finally, the result to determine the sign of discriminant is useful:
\begin{lemma}
Let $K$ be an algebraic number field, then $$d(K)=(-1)^{r_2}|d(K)|$$
\end{lemma}

\section{Decomposition of Prime Numbers}
For simplicity, we continue to work with a number field $K$ considered as an (finite) extension of $\mathbb{Q}$, and not considered as a relative extension. Many of the results which are explained in that context are still true in the more general case, but some are not. Almost always, these generalizations fail because the ring of integers of the base field is not a PID(Dedekind).
The main results concerning the decomposition of primes are as follows:
\begin{proposition}\label{thm:decomposition}
Let $p$ be a prime number, then there exist positive integers $e_i$ such that $$pO_K=\prod_{i=1}^g \wp_i^{e_i},$$ where $\wp_i$ are all the prime ideals above $p$, i.e. $\wp_i\cap \mathbb{Z}=p\mathbb{Z}$.
\end{proposition}

The integer $e_i$ is called the \textbf{ramification index} of $p$ at $\wp_i$ and is denoted $e(\wp_i|p)$. The degree $f_i$ of the field extension defined by $$f_i=[O_K/\wp_i:\mathbb{Z}/p\mathbb{Z}]$$ is called the \textbf{residue degree} of $p$ and is denoted $f(\wp_i|p)$. $g$ is called the \textbf{decomposition number} of $p$ in $K$.

There is an important relation between these coefficients, which comes to a theorem.
\begin{proposition}
Let $[K:\mathbb{Q}]=n$, then for any $p$, the decomposition in Theorem  \ref{thm:decomposition} satisfies $$\sum_{i=1}^g e_if_i=n.$$
\end{proposition}

Let $pO_K=\prod_{i=1}^g \wp_i^{e_i}$ be the decomposition of a prime $p$. We will say that $p$ is \textbf{inert} if $g=1$ and $e_i=1$,i.e. $pO_K=\wp_i$.
We will say that $p$ \textbf{splits completely} if $g=n$.
Finally, we say that $p$ is \textbf{ramified} if there is an $e_i$ which is greater than or equal to 2 (in other words if $pO_K$ is not squarefree), otherwise we say that $p$ is \textbf{unramified}. Those prime ideals $\wp_i$ such that $e_i>1$ are called the ramified prime ideals of $O_K$. In particular, if $e_1=n$, then we say that $p$ \textbf{ramifies totally}.

From the definitions of these ramification index and decomposition number satisfy chain rule, which is a very important message for us to decompose prime number in a compositum of fields. 

In the case when $K/\mathbb{Q}$ is a Galois extension, the result is more specific:
Assume $K/\mathbb{Q}$ is a Galois extension. Then for any $p$, the ramification indices $e_i$ are equal, the residual degrees $f_i$ are equal as well, hence $e f g=n$. 
In addition, the Galois group operates \textbf{transitively} on the prime ideals above $p$: i.e. there exists $\sigma\in\operatorname{Gal}(K)$, such that $\sigma(\wp_i)=\wp_j$.


The existence of ramified prime has showed by Minkowski: If $K$ is a number field different from $\mathbb{Q}$, then $|d(K)|>1$. In particular, there exists at least one ramified prime in $K$. What's more, the fundamental ramification theorem\citep{cohen1993course} is as follows:
\begin{proposition}\label{thm:ramification}
Let $p$ be a prime number, then $p$ is ramified in $K$ if and only if $p$ divides the discriminant $d(K)$. In particular, there are only a finite number of ramified primes (exactly $w(d(K))$, where $w(x)$ is the number of distinct prime divisors of an integer $x$).
\end{proposition}

On the contrary, for unramified prime, we have another theorem given by Stickelberger\citep{cohen1993course}.
\begin{proposition}[Stickelbeger]\label{thm:unramified}
If $p$ is an unramified prime in $K$ with $pO_K=\prod_{i=1}^g \wp_i$, we have $$\left(\frac{d(K)}{p}\right)=(-1)^{n-g},$$ for $p=2$, $\left(\frac{d(K)}{2}\right)=(-1)^{n-g}$ is to be seen as the Jacobi-Kronecker symbol(See Appendix \ref{chap:appA}). 
\end{proposition}

%\begin{corollary}[quadratic field]
%The decomposition type of a prime number $p$ in a quadratic field $K$ of discriminant $D$ is the %following: if $\left(\frac{D}{p}\right)=-1$, then $p$ is inert. If  $\left(\frac{D}{p}\right)=0$, then $p$ is ramified. Finally, if $\left(\frac{D}{p}\right)=1$, then $p$ splits(completely).
%\end{corollary}

We now consider a more difficult algorithmic problem, that of determining the decomposition of prime numbers in a number field. The basic theorem on the subject, which unfortunately is not completely sufficient(but right for Dedekind domain with power integral basis, even for without essential factor), is as follows.

\begin{proposition}[Kummer]\label{thm:kummer}
Let $K=\mathbb{Q}(\theta)$ be a number field, where $\theta$ is an algebraic integer, whose minimal polynomial is denoted $T(x)$. Let $f$ be the index of $\theta$, i.e. from definition $f=[O_K:\mathbb{Z}[\theta]]$. Then for any prime $p$ not dividing $f$ on can obtain the prime decomposition of $pO_K$ as follows. Assume $$T(x)\equiv \prod_{i=1}^g T_{i}(x)^{e_i} (\operatorname{mod } p)$$ be the decomposition of $T$ into irreducible factors in $\mathbb{F}_p[x]$, where the $T_i$ are also monic. Then $$pO_K=\prod_{i=1}^g\wp_i^{e_i},$$ where $$\wp_i=(p,T_i(\theta))=pO_K+T_i(\theta)O_K$$
Furthermore, the residual index $f_i$ is equal to the degree of $T_i$. 
\end{proposition}

\section{Units and Ideal Classes}\label{sec:unitsideal}
Let $K$ be a number field and $O_K$ be the ring of integers of $K$. We say that two (fractional) ideals\footnote{Fractional ideal $I$ in $O_K$ is a non-zero sub-module of $K$ such that there exists a non-zero integer $d$ with $d I$ ideal of $O_K$.} $I$ and $J$ of $K$ are equivalent if there exists $\alpha\in K^*$ such that $J=\alpha I$. The set of equivalence classes is called the \textbf{class group} of $O_K$ and is denoted $Cl(K)$. 

Since fractional ideals of $O_K$ form a group it follows that $Cl(K)$ is also a group. The main theorem concerning $Cl(K)$ is that it is finite.

For any number field $K$, the class group $Cl(K)$ is a finite Abelian group, whose cardinality, called the \textbf{class number}, is denoted $h(K)$. Note that $h(K)=1$ if and only if $O_K$ is a PID(UFD).

Denote by $I(K)$ the set of fractional ideals of $K$, and $P(K)$ the set of principal ideals. We clearly have the exact sequence:
$$1\rightarrow P(K)\rightarrow I(K)\rightarrow Cl(K)\rightarrow1.$$

The set of units in $K$ form a multiplicative group which we will denote by $U(K)$. Units are algebraic  integers of norm equal to $\pm1$. The torsion subgroup of $U(K)$, i.e. the group of roots of unity in K, will be denoted by $\mu(K)$.

It is clear that we have the exact sequence:
$$1\rightarrow U(K)\rightarrow K^{\times}\rightarrow P(K)\rightarrow1.$$
To sum up above two sequence, we have new exact sequence as follows:
$$1\rightarrow U(K)\rightarrow K^{\times}\rightarrow P(K)\rightarrow I(K)\rightarrow Cl(K)\rightarrow1.$$

The main result concerning units is the following theorem:
\begin{proposition}[Dirichlet's Unit Theorem]\label{thm:Dirichlet}
Let $(r_1,r_2)$ be the signature of $K$, then $U(K)$ is finitely generated Abelian group of rank $r_1+r_2-1$. i.e. we have a group isomorphism: $$U(K)\cong\mu(K)\times\mathbb{Z}^{r_1+r_2-1},$$ and $\mu(K)$ is a finite cyclic group.
\end{proposition}
If we set $r = r_1+r_2-1$, we see that there exist units $u_1,\dots,u_r$ such that every element $x$ of $U(K)$ can be written in a unique way as $$x=\zeta u_1^{n_1}\cdots u_r^{n_r},$$ where $n_i\in\mathbb{Z}$ and $\zeta$ is a root of unity in $K$.Such a family $(u_i)$ is called a system of \textbf{fundamental units} of $K$. 

A very important property is the number is finite, what's more, like we claimed before, the ideal class group for any number field is a finite Abelian group. As for the class number, we can give a certain upper bound of it. First of all, we give the definition of Minkowski bound \citep{mollin1999algebraic}:
\begin{definition}\label{def:minkowski}
If $K$ is a number field, the quantity $$C_K=\left(\frac{4}{\pi}\right)^{r_2}\frac{n!}{n^n}\sqrt{|d(K)|}$$ is called the  \textbf{Minkowski's bound}, where $d(K)$ is the discriminant of $K$ and $[K:\mathbb{Q}]=n$ with signature $\{r_1,r_2\}$.
\end{definition}
The following lemma \citep{mollin1999algebraic} will give us a useful method to determined the class number and class field for $n=[K:\mathbb{Q}]$ is small.
\begin{lemma}
Any ideal class of a number field $K$ has an (integral) ideal $I$, such that $$\operatorname{N}(I)\leq C_K$$
\end{lemma}
From this lemma, we can also get the so-called Hermite's theorem on discriminant: There are only finitely many number fields having a given discriminant $d$.

In fact, we have a following method to determined the class number and class group" Firstly, we should calculated the Minkowski's bound of the number field $K$. From the lemma, we can find all rational primes that $p\leq C_K$. Given the prime decomposition of $pO_K$, then we can compute all prime ideals $\wp$ over $p$. Hence the class group $Cl(K)$ is generated by $A=\{[\wp]|\wp|p\leq C_K\}$, where $[\wp]$ is the ideal class which $\wp$ lies in. If $A$ is not so big, the we can consider the multiple relationship between its elements, then we can get the class group and class number.

\section{Character and Conductor}
A \textbf{character} on a group $G$ is a group homomorphism from $G$ to the multiplicative group of a field( usually complex numbers field). The set $\hat{G}$ of these morphisms forms an abelian group under pointwise multiplication. Sometimes we only consider unitary characters, thus the image is in the unit circle.

Now we consider the special character we mentioned above, \textbf{Dirichlet character}, which is defined as follows:
\begin{definition}
A Dirichlet character is any function $\chi$ from the integers $\mathbb{Z}$ to the complex numbers $\mathbb{C}$ such that $\chi$ has the following properties:
\begin{enumerate}
\item the function is periodic, i.e. $\exists k\in\mathbb{Z^+}$, s.t. $\chi(n)=\chi(n+k),\forall n$.
\item If $\operatorname{gcd}(n,k)>1$, then $\chi(n)=0$; if $\operatorname{gcd}(n,k)=1$, then $\chi(n)\neq0$
\item $\chi(mn)=\chi(m)\chi(n)$ for all integers $m,n$.
\end{enumerate}
\end{definition}
The Dirichlet character has following properties: from the definition, we can directly get $\chi(1)=1$, the character is periodic with period $k$, we say that $\chi$ is a character to the \textbf{modulus} $k$. i.e. we have $$a\equiv b \mod k \Rightarrow \chi(a)=\chi(b)$$
If $\operatorname{gcd}(a,k)=1$, then from Euler theorem, we have $a^{\phi(k)}\equiv 1 \mod k$, therefore we have $\chi(a^{\phi(k)})=\chi(1)=1$, on the other hand, $\chi(a^{\phi(k)})=\chi(a)^{\phi(k)}$. i.e. for all $a$ relatively prime to $k$, $\chi(a)$ is a $\phi(k)$-th complex root of unity. 

A character $\chi$ is said to be \textbf{odd} if $\chi(-1)=-1$ and \textbf{even} if $\chi(-1)=1$. A character is called \textbf{principal} if it assumes the value $1$ for arguments coprime to its modulus and otherwise is 0. A character of the (Abelian) field can be viewed as the character of the Galois group of the field.

For example, the character of a quadratic field $K$ is $\hat{K}=\{1,\chi\}$ (see \citep{Xianke2006ANT} etc.). the character $\chi$ is the same to the Legendre-Kronecker symbol (See Appendix \ref{chap:appA}).

Then, we will introduce the conductor. Taking as base the field of rational numbers, the Kronecker–Weber theorem \ref{thm:kronecker} states that an algebraic number field $K$ is abelian over $\mathbb{Q}$ if and only if it is a subfield of a cyclotomic field $\mathbb{Q}(\zeta_n)$. The \textbf{conductor} of $K$ is then the smallest such $n$. 

we can also define a conductor of character, i.e. the conductor of a character is the smallest modulus of $\chi$, More precisely, the conductor of a Dirichlet character $\chi$ modulo $k$ is the smallest positive integer $k_0$ which divides $k$ and which has the property that $\chi(n+k_0)=\chi(n)$ for all $n$. For this case, the $\chi$ is called the \textbf{primitive character of conductor} (or modulus) $f_{\chi}$. 

The relation between the conductor of characters and Abelian number field is that, the conductor $f$ of the field $K$ is the least common multiple of conductor of character for Galois group of the Abelian number field, i.e. $$f=\operatorname{lcm}_{\chi\in\hat{K}}\{f_{\chi}\}.$$

For example, for real quadratic field, $f$ is the fundamental discriminant of the field. For cyclic cubic field, $f$ is actually $e$ in Theorem \ref{thm:ccpoly}, i.e. the arithmetic square root of the discriminant of the cyclic cubic field.

Let $p$ be an odd prime and $K/\mathbb{Q}$ a cyclic extension of degree $p$. Then it is well known \citep{maki1980determination} that the conductor of $K$ must have the form $f=p^e\cdot q_1q_2\cdot q_n$, where $e=0$ or $2$, $n\geq0$, and the $q_i$ are pairwise distinct rational primes satisfying $q_i\equiv1(\operatorname{mod} p)$ for $i=1,2,\dots,n$. The discriminant of $K$ is just a power of the conductor, $d_K=f^{p-1}$.

A theorem called \textbf{conductor-discriminant formula} related to the conductor of a field and the discriminant of the field was first found by Dedekind. Then at the beginning of 1930's, E. Artin and H. Hasse found this general formula (See \citep{rzedowski2011conductor} etc.) which is showed following:
\begin{proposition}\label{thm:conddisc}
Let $K/F$ be a finite Galois extension of global fields with Galois group $G$, $$d(K/F)=\prod_{\chi\in\hat{G}}f_{\chi}^{\chi(1)},$$
since for abelian field, $\chi(1)=1$, hence for $K$ is an abelian number field, then we have a special form, $$d(K)=(-1)^{r_2}\prod_{\chi\in\hat{G}}f_{\chi}$$
\end{proposition}


  \chapter{Quadratic Fields}
\label{chap:chap-four}
In this chapter, we are going to discover the most simplest number field that are different from
$\mathbb{Q}$, i.e. quadratic fields. Let denote it as $K$. A quadratic field is of degree 2 over $\mathbb{Q}$, it can be given by $K=\mathbb{\theta}$, where $\theta$ is a root of minimal polynomial $f(x)=x^2+a x+b$ of $\mathbb{Z}[x]$. Let $d=a^2-b$, then $K=\mathbb{Q}(\sqrt{d})$. Clearly, $d$ is not a square, otherwise $f(x)$ won't be a irreducible quadratic polynomial. Since $\mathbb{Q}(\sqrt{m^2d})=\mathbb{Q}(\sqrt{d})$, hence we may assume that $d$ is squarefree.

Since $n=2=r_1+2r_2$, the signature of quadratic field $K$ is either $(2,0)$ or $(0,1)$, which is called real quadratic field or complex (or imaginary) quadratic field respectively.
It's easy to show that the real quadratic field has positive discriminant and the complex quadratic field has negative discriminant.

\section{Discriminant, Integral Basis}
The integral basis and discriminant of quadratic field is easy as following theorem:
\begin{theorem}\label{thm:quad_disc}
Let $K=\mathbb{Q}(\sqrt{d})$ be a quadratic field with $d$ squarefree. Then if $d\equiv1(\operatorname{mod} 4)$, we have integral basis $(1,(1+\sqrt{d})/2)$ with discriminant $d(K)=d$; if $d\equiv2\text{ or }3(\operatorname{mod} 4)$, we have integral basis $(1,\sqrt{d})$ with discriminant $d(K)=4d$.
\end{theorem}

Let us denote $D=d(K)$ (namely fundamental discriminant) for these two cases, then we can see that it satisfies Stickelberger's criterion \ref{thm:stickelberger}. What's more, $K=\mathbb{Q} (\sqrt{D})$ has integral basis $(1,\omega)$, where $$\omega=\frac{D+\sqrt{D}}{2},$$ and therefore $O_K=\mathbb{Z}[\omega]$.

\section{Decomposition of Primes}
Note that Proposition \ref{thm:decomposition} and Proposition \ref{thm:kummer} immediately shows how prime numbers decompose in a quadratic field \citep{Xianke2006ANT}.


\begin{theorem}\label{thm:quad_decompose}
Let $K=\mathbb{Q}(\sqrt{D})$, where $D$ is the fundamental discriminant, i.e. $D=d(K)$, $O_K=\mathbb{Z}[\omega]$ where $\omega=(D+\sqrt{D})/2$ its ring of integers, and $p$ be a prime number. Then\footnote{$\left(\frac{D}{p}\right)$ is Kronecker symbol which can be seen in Appendix \ref{chap:appA}.}
\begin{enumerate}
\item If $\left(\frac{D}{p}\right)=0$, then $p$ is ramified, i.e. $pO_K=\wp^2$. More precisely, $$\wp=pO_K+\omega O_K,$$ except when $p=2$ and $D\equiv12(\operatorname{mod} 16)$.
\item If $\left(\frac{D}{p}\right)=-1$, then $p$ is inert, hence $pO_K=\wp$ is a prime ideal.
\item If $\left(\frac{D}{p}\right)=1$, then $p$ is split, and we have $pO_K=\wp_1\wp_2$, where 
$$\wp_i=pO_K+(\omega-\frac{D\pm b}{2})O_K,$$ and $b$ is any solution to $b^2\equiv D(\operatorname{mod} 4p)$
\end{enumerate}
\end{theorem}

\section{Unit Group of Imaginary Quadratic Fields}
First we consider the unit group of quadratic field, by the Dirichlet's Unit theorem \ref{thm:Dirichlet}, we can divide the quadratic field into two cases. The imaginary quadratic field has simple unit group since $r=r_1+r_2-1=0$, i.e. the unit group is finite. In fact, the imaginary quadratic field is the only number field apart from $\mathbb{Q}$ who has finite units. A theorem for the unit group of Imaginary Quadratic fields could be found in \citep{Xianke2006ANT,cohen1993course} etc.:

\begin{theorem}\label{thm:quad_unit}
Let $K=\sqrt{D}$, where the fundamental discriminant $D<0$, then the group $U(K)=\mu(K)$ of units  is equal to the group of $\omega(D)-th$ roots of unity, where 
$$\omega(D)=\left\{\begin{array}{ll}
2,& \text{if }D<-4\\
4,& \text{if }D=-4\\
6,& \text{if }D=-3
\end{array}\right.$$
\end{theorem}

\section{Class number of Imaginary Quadratic Fields}
Let us now consider the problem of computing the class number of imaginary quadratic field. Here we give a beautiful result from L-function. Since it's to far to enter into the details of the analytic theory of \textbf{L-functions}, so we just recall the results. Our main result is a corollary of Dirichlet's Theorem (We have reorganized this theorem to fit all imaginary quadratic number field. In the original theorem, it just states $D<-4$ cases.) \citep{cohen1993course}.
 
\begin{theorem}
If $D$ is a negative fundamental discriminant, then $$h(K)=\frac{\omega(D)}{4-2\left(\frac{D}{2}\right)}\sum_{1\leq r<|D|/2}\left(\frac{D}{r}\right),$$ where $\omega(D)$ is defined in Theorem \ref{thm:quad_unit}. 
\end{theorem} 

\begin{remark}
For $D<-4$, we have $$h(K)=\frac{1}{2-\left(\frac{D}{2}\right)}\sum_{1\leq r<|D|/2}\left(\frac{D}{r}\right).$$ Of course, we can get $h(-3)=h(-4)=1$ by calculation. 
\end{remark}
More precise list for class number can be seen in the book \citep{Xianke2006ANT}.

\section{Unit group of Real Quadratic Fields}
For real quadratic fields, ZHANG \citep{Xianke2006ANT} has given a precise algorithm to compute it through the Pell's equation. Since the unit circle of a real quadratic field only hits two time real axis, then we have $\mu{K}=\{\pm1\}$. By Dirichlet Unit Theorem \ref{thm:Dirichlet}, we have $$U(K)=\{\pm1\}\times \epsilon^{\mathbb{Z}},$$ where $\epsilon$ generates an infinite cyclic group $\langle\epsilon\rangle=\epsilon^{\mathbb{Z}}$. There exists only one unit in the set (all of them can generate whole infinite cyclic group) $\{\pm\epsilon,\pm\epsilon^{-1}\}$ which is larger than 1, and we denote it as fundamental unit of $K$. 
Now we just need to compute the fundamental unit. Let $K=\mathbb{Q}(\sqrt{d})$, where $d$ is squarefree. Let $\alpha=a+b\sqrt{d}(a,b\in\mathbb{Q})$ be unit of $K$, then $N(\alpha)=\pm1$. 
If $d\equiv 2 \text{ or }3 \mod 4$, then $O_K=\mathbb{Z}[\sqrt{d}]$. A integer $\alpha=x+y\sqrt{d}(x,y\in\mathbb{Z})$ is unit if and only if $N(\alpha)=\pm1$, i.e. we have following Pell's equation:
\begin{equation}\label{eqn:pell1}
x^2-d y^2=\pm1
\end{equation} 
From the Unit Theorem, we note that if $\epsilon=a_1+b_1\sqrt{d}$ is a fundamental unit of $K$($a_1,b_1>0$), then $$\epsilon^n=(a_1+b_1\sqrt{d})^n=a_n+b_n\sqrt{d}$$ is also a unit who larger than one. What's more, $(a_n,b_n)$ are natural number solutions of the Pell's equation \ref{eqn:pell1}. Note that $b_{n+1}=a_1b_n+a_nb_1$, hence $b_n$ is an increasing sequence.

Let us consider $b=1,2,3,\cdots$, if $d b^2\pm1$ is a square number (i.e. $a^2$), then we stop the process, i.e. we find the "smallest" solution of the Pell's equation \ref{eqn:pell1} and this solution is also the fundamental unit.

Let's consider $d=6$, $6b^2\pm1$ is a square number firstly when $b=2$, i.e. we have the fundamental unit of $\mathbb{Q}(\sqrt{6})$ is $\epsilon=5+2\sqrt{6}$.

Similarly, if $d\equiv1 \mod 4$, then $O_K=\mathbb{Z}[\frac{1+\sqrt{d}}{2}]$, an integer of $K$ has form $\alpha=(a+b\sqrt{d})/2$, where $a\equiv b \mod 2$, $a,b\in\mathbb{Z}$. $\alpha$ be unit of $K$ if and only if $N(\alpha)=(a^2-b^2d)/4=\pm1$, hence, $(a,b)$ is a solution of the following Pell's equation:
\begin{equation}\label{eqn:pell2}
x^2-d y^2=\pm4
\end{equation}
if $\epsilon=(a_1+b_1\sqrt{d})/2$ is a fundamental unit of $K$($a_1,b_1>0$), then $$\epsilon^n=(\frac{a_1+b_1\sqrt{d}}{2})^n=\frac{a_n+b_n\sqrt{d}}{2}$$ is also a unit who larger than one. Similarly, consider $b=1,2,3,\cdots$, if $d b^2\pm4$ is a square number (i.e. $a^2$), then we stop the process and get the fundamental unit $(a+b\sqrt{d})/2$.

\section{Class Numbers of Real Quadratic Fields}
Also, as in the imaginary case, using L-function, we can also get a beautiful results\citep{cohen1993course}. We have modified the results through discarding regulators.

\begin{theorem}\label{thm:rquad_cl}
If $D$ is a positive fundamental discriminant, then $$h(K)=-\frac{1}{\operatorname{ln}(\epsilon)}\sum_{r=1}^{\lfloor(D-1)/2\rfloor}\left(\frac{D}{r}\right)\operatorname{ln}\sin\left(\frac{r\pi}{D}\right).$$
\end{theorem}

Now we see an example, let $D=4d=8$, then $\epsilon=1+\sqrt{2}$ by the result of unit group of real quadratic field. From the Theorem \ref{thm:rquad_cl}, we have $$h(K)=-\frac{1}{\operatorname{ln}(1+\sqrt{2})}\left(\operatorname{ln}\sin(\frac{\pi}{8})-\operatorname{ln}\sin(\frac{3\pi}{8})\right)=-\frac{\operatorname{ln}(\tan(\frac{\pi}{8}))}{\operatorname{ln}(1+\sqrt{2})}=1.$$

 
  \chapter{Cyclic Cubic Fields}
\label{chap:chap-five}
In this chapter, we start with a cubic polynomial with Galois group $A_3=C_3$. A famous work with only half page given by Seidelmann in 1917 \citep{seidelmann1917gesamtheit} showed the condition of the polynomial which satisfies the Galois group $C_3$. We have recovered the process to get this condition with a lemma which can be found in Cohen's book. Then we proceed to rearrange the explicit results for cyclic cubic field based on Cohen's book. At last, we will propose some examples on prime decomposition and computing class number.

In additional, we have offered some special examples in the last section \ref{sec:ssccf} for cyclic cubic field without the standard form which we will show as follows. What's more, we also give a formula (See theorem \ref{thm:ccuformula}) for a type of cyclic cubic field which is given risen by the ideas of F.C. Orvay's \citep{orvay1991cyclic}.

Firstly, we propose some preliminaries of cyclyc cubic field $K$. Let $K$ be a number field of degree 3 over $\mathbb{Q}$, i.e. a cubic field. If $K$ is Galois over $Q$, with $\operatorname{Gal}(K)=A_3$, then $K$ is so called a cyclic cubic field. let denote the Galois group of $K$ be $\langle\sigma\rangle$, where $\sigma^{-1}=\sigma^2$. Note that, for cyclic cubic field $K$ can only be totally real, based on the Lemma \ref{lem:signaturegal}.

\section{Cubic Polynomial and Cyclic Cubic Field}
The cyclic cubic fields can be viewed as generating by cubic polynomials. let's consider a polynomial of degree 3 which has the form
$$f(x)=ax^3+bx^2+cx+d$$
where $a\neq0$. There are two possible Galois groups for splitting field of cubic polynomial, namely $S_3$ and $A_3=C_3$.

Now we will reduce the general cubic equation into a cubic trinomial by eliminating the quadratic term. We begin with the general cubic with rational coefficients $$ax^3+bx^2+cx+d$$
and make the substitution $x=X-\frac{b}{3a}$ to get
$$aX^3+\left(c-\frac{b^2}{3a}\right)X-\frac{bc}{3a}+d+\frac{2b^3}{27a^2}$$
Since each of these coefficients are rational, we have now have a cubic trinomial in $\mathbb{Q}$. Also, since we are working over the rational numbers, we can easily divide by the leading coefficient of the $x^3$ term and obtain a monic cubic equation.

The main distinction of cubics with a Galois group of $C_3$ is that the polynomial discriminant is equal to a square in $\mathbb{Q}$ which is a direct corollary of Theorem \ref{thm:discriminantsq}. Seidelmann uses this fact to give a form for the coefficients of a monic cubic trinomial with a Galois group of $C_3$. For $p,q\in\mathbb{Q}$, the equation
$$f(x)=x^3-3(p^2+3q^2)x+2p(p^2+3q^2)$$
where $f(x)$ is not reducible, represents all equations of degree 3 with a Galois group of $C_3$.

In Seidelmann's paper (German version), there are only few rows to show the result but without proof, so now we give a brief proof of it. We assume that $K$ is a cyclic cubic field. Let $\theta$ be an algebraic integer such that $K=\mathbb{Q}(\theta)$, and let $f(x)=x^3+ax+b$ be the minimal polynomial of $\theta$ as we mentioned before. 

One important consequence is necessary for calculation. Since any cyclic cubic field has at least one real embedding and since $K$ is Galois, all the roots of $f$ must be real(See Lemma \ref{lem:signaturegal}). Of course, we can get this from the square discriminant of $f$.

Consider the primitive cube roots of unity $\zeta=e^{2\pi i/3}$, then it's easy to see that $K(\zeta)$ is a sextic field over $\mathbb{Q}$. What's more, it's also Galois with Galois group $\langle\sigma,\tau\rangle$, where $\sigma$ has been defined above which fixes $\zeta$, $\tau$ is the complex conjugation. We first prove the following lemma, one could find a similar lemma in Cohen's book  \citep{cohen1993course}.

\begin{lemma}
Set $\gamma=\theta+\zeta^2\sigma(\theta)+\zeta\sigma^2(\theta)\in K(\zeta)$, and $\beta=\gamma^2/\tau(\gamma)$. Then $\beta\in\mathbb{Q}(\zeta)$ and we have $$f(x)=x^3-\frac{e}{3}x-\frac{eu}{27},$$ where $e=\beta\tau(\beta)$ and $u=\beta+\tau(\beta)$. 
\begin{proof}
We have $\tau(\gamma)=\theta+\zeta\sigma(\theta)+\zeta^2\sigma^2(\theta)$, also we can verify that $$\sigma(\gamma)=\sigma(\theta)+\zeta^2\sigma^2(\theta)+\zeta\theta=\zeta\gamma$$ and $$\sigma(\tau(\gamma))=\sigma(\theta)+\zeta\sigma^2(\theta)+\zeta^2\theta=\zeta^2\tau(\gamma).$$ Hence, we have $$\sigma(\beta)=\sigma(\gamma^2/\tau(\gamma))=\gamma^2/\tau(\gamma)=\beta,$$ i.e. $\beta$ is invariant under the action of $\sigma$, so by the Galois theory $\beta\in\mathbb{Q}(\zeta)$.

Note that $e$ and $u$ are norm and trace of $\beta$ in $\mathbb{Q}(\zeta)$ respectively, hence is rational numbers.

Now we have the following matrix equation: 
\begin{equation*}
\left(\begin{array}{c}0\\ \gamma\\ \tau(\gamma)\end{array}\right)=\left(\begin{array}{ccc}1&1&1\\ 1&\zeta^2&\zeta\\ 1&\zeta&\zeta^2\end{array}\right)\left(\begin{array}{c}\theta\\ \sigma(\theta)\\ \sigma^2(\theta)\end{array}\right)
\end{equation*}
Then, we have 
\begin{equation*}
\left(\begin{array}{c}\theta\\ \sigma(\theta)\\ \sigma^2(\theta)\end{array}\right)=\frac{1}{3}\left(\begin{array}{ccc}1&1&1\\ 1&\zeta&\zeta^2\\ 1&\zeta^2&\zeta\end{array}\right)\left(\begin{array}{c}0\\ \gamma\\ \tau(\gamma)\end{array}\right)
\end{equation*}
From the formula, we have $$a=\theta\sigma(\theta)+\theta\sigma^2(\theta)+\sigma(\theta)\sigma^2(\theta)=-\frac{\gamma\tau(\gamma)}{3}$$ and $$b=-\theta\sigma(\theta)\sigma^2(\theta)=-\frac{\gamma^3+(\tau(\gamma))^3}{27}.$$
Note that $\tau(\beta)=\tau(\gamma^2/\tau(\gamma))=\tau(\gamma^2)/\gamma=(\tau(\gamma))^2/\gamma$. We can easily verify that the norm and trace of $\beta$ coincide $e$ and $u$.
\end{proof}
\end{lemma}

Now since $\mathbb{Q}(\zeta)=\mathbb{Q}(\sqrt{-3})$, we can assume that $\beta=p+q\sqrt{-3}\in\mathbb{Q}(\zeta)$, where $p,q\in\mathbb{Q}$ then $e=\operatorname{N}(\beta)=(p+q\sqrt{-3})(p-q\sqrt{-3})=p^2+3q^2$, $u=\operatorname{Tr}(\beta)=2p$, hence we have 
$$f(x)=x^3-\frac{p^2+3q^2}{3}x-\frac{2p(p^2+3q^2)}{27},$$
which is equivalent to $f(x)=x^3-3(p^2+3q^2)x-2p(p^2+3q^2)=x^3-3ex-eu$ up to scaling. (The sign of the constant term's coefficient can be change by replace $p$ into $-p$.)

\section{Discriminant and Integral Basis}
Now, for simplicity, up to suitable scaling, we use $f(x)=x^3-3ex-e u$, where wlog, we assume that $e,u$ are rational integer. Note that then $\beta$ can be written down an algebraic integer of $\mathbb{Q}(\zeta)$,i.e. we have $\beta=\frac{u+v\sqrt{-3}}{2}, u,v\in\mathbb{Z}$. What's more, $u$ cannot be divisible by 3 since $\beta$ is not divisible by the ramified prime. Hence, by suitable choosing $\beta$ or $-\beta$, we may assume that $u\equiv2(\operatorname{mod} 3).$
In this notation, $e=\frac{u^2+3v^2}{4}$, then $e\equiv 1(\operatorname{mod} 3)$. In fact, $e$ is the product of distinct primes which congruent to 1 modulo 3 (including $e=1$) \citep{cohen1993course}.
i.e. we have following lemma:
\begin{lemma}\label{basiccubic}
For any cyclic cubic field $K$, there exists a unique pair of integer $e,u$ such that $e$ is equal to a product of distinct primes congruent to 1 modulo 3, $u\equiv2(\operatorname{mod} 3)$, $v>0$ and such that $K=\mathbb{Q}(\theta)$, where $\theta$ is a root of the polynomial $$f(x)=x^3-3ex-eu.$$
Moreover, the conjugates of $\theta$ are given by following formulas:
\begin{eqnarray}\label{eqn:conjugatescub}
\sigma(\theta)&=&\frac{-2e}{v}-\frac{u+v}{2v}\theta+\frac{\theta^2}{v},\\
\sigma^2(\theta)&=&\frac{2e}{v}+\frac{u-v}{2v}\theta-\frac{\theta^2}{v}
\end{eqnarray}
\begin{proof}
For simplicity, we just give the proof of the conjugates. Firstly, since the discriminant of $x^3+ax+b$ is equal to $-(4a^3+27b^2)$ (Also can refer the Appendix \ref{prop:disctri}), hence the discriminant of $f$ is equal to $$-(4(-3e)^3+27e^2u^2)=-27e^2(u^2-4e)=81e^2v^2.$$
Then from the definition of Polynomial we have $d=(\theta-\sigma(\theta))(\sigma(\theta)-\sigma^2(\theta))(\sigma^2(\theta)-\theta)=\pm9ev$. If necessary, by exchanging the $\sigma(\theta), \sigma^2(\theta)$, we may assume that $$\sigma(\theta)-\sigma^2(\theta)=\frac{9ev}{(\theta-\sigma(\theta))(\theta-\sigma^2(\theta))}=\frac{9ev}{f'(\theta)}=9ev/(3\theta^2-3e).$$
Now we should simplify this equation. Since $f(\theta)=\theta^3-3e\theta-eu=0$, hence we can use the extended Euclidean algorithm with $A(x)=x^3-3ex-eu$ and $B(x)=x^2-e$, then we got the inverse of $B$ modulo $A$ is equal to $(2x^2-ux-4e)/(3v^2e)$ (note that $\operatorname{gcd}(A,B)=1$ for there is no multiple roots in $f(x)$.), hence $$\sigma(\theta)-\sigma^2(\theta)=\frac{2\theta^2-u\theta-4e}{v}$$
On the other hand, since the trace of $\theta$ is equal to 0, so we have $\sigma(\theta)+\sigma^2(\theta)=-\theta$, we can get the final result of $\sigma(\theta)$ and $\sigma^2(\theta)$ from above equations.
\end{proof}
\end{lemma}

The following theorem\citep{cohen1993course} shows the integral basis and discriminant of cyclic cubic field:
\begin{theorem}\label{thm:cbfintegralnor}
Let $K=\mathbb{Q}(\theta)$ be a cyclic cubic field where $\theta$ is a root of $x^3-3ex-eu=0$ and where, as above, $e=\frac{u^3+3v^2}{4}$ is equal to a product of distinct primes (namely $t$ distinct primes) congruent to 1 modulo 3, $u\equiv2(\operatorname{mod} 3)$, then
\begin{enumerate}
\item Assume that $3\nmid v$, i.e. 3 is ramified in $K$. Then $(1,\theta,\sigma(\theta))$\footnote{$\sigma$ is given by lemma \ref{basiccubic}.} is an integral basis of $K$ and the discriminant of $K$ is equal to $(9e)^2$. What's more, there exists up to isomorphism exactly $2^t$ cyclic cubic fields of discriminant $(9e)^2$ defined by the polynomial.
\item Assume that $3\mid v$, i.e. 3 is unramified in $K$. Then let $\theta'=(\theta+1)/3$, $(1,\theta',\sigma(\theta'))$ is an integral basis of $K$ and the discriminant of $K$ is equal to $e^2$. What's more, there exists up to isomorphism exactly $2^{t-1}$ cyclic cubic fields of discriminant $e^2$ defined by the polynomial.
\end{enumerate}
\end{theorem}

For the first case, not that $\theta^2=v\sigma(\theta)+((u+v)/2)\theta+2e$ from the result of $\sigma(\theta)$, so the $\mathbb{Z}$-module $O_K$ generated by $(1,\theta,\sigma(\theta))$ contains $\mathbb{Z}[\theta]$. So the index $[O_K:\mathbb{Z}[\theta]]=v$.
For the second case, ie, if the prime 3 is unramified in $K$, then we can write the minimal polynomial of $\theta'$, i.e. $\theta'$ is a root of the equation with coefficients in $\mathbb{Z}$ \begin{equation} f(x)=x^3-x^2+\frac{1-e}{3}x-\frac{1-3e+eu}{27},\end{equation} where  $e=\frac{u^2+27v'^2}{4},u\equiv2(\operatorname{mod} 3), u\equiv v' (\operatorname{mod} 2), v'>0$, and $e$ is the product of distinct primes congruent to 1 modulo 3. For the same reason, we have $[O_K:\mathbb{Z}[\theta']]=v/3=v'$.


With new notation on $e,u,v$, Cohen's book\citep{cohen1993course} given us another theorem as follows:
\begin{theorem}\label{thm:ccpoly}
All cyclic cubic fields $K$ are given exactly once (up to isomorphism) in the following way:
\begin{enumerate}
\item If the prime 3 is ramified in $K$, then $K=\mathbb{Q}(\theta)$ where $\theta$ is a root of the equation with coefficients in $\mathbb{Z}$ \begin{equation} f(x)=x^3-\frac{e}{3}x-\frac{e u}{27},\end{equation} where $e=\frac{u^2+27v^2}{4},u\equiv6(\operatorname{mod} 9),3\nmid v, u\equiv v (\operatorname{mod} 2), v>0$, and $e/9$ is the product of distinct primes congruent to 1 modulo 3 (could be 1).
\item If the prime 3 is unramified in $K$, then $K=\mathbb{Q}(\theta)$ where $\theta$ is a root of the equation with coefficients in $\mathbb{Z}$ \begin{equation} f(x)=x^3-x^2+\frac{1-e}{3}x-\frac{1-3e+eu}{27},\end{equation} where  $e=\frac{u^2+27v^2}{4},u\equiv2(\operatorname{mod} 3), u\equiv v (\operatorname{mod} 2), v>0$, and $e$ is the product of distinct primes congruent to 1 modulo 3.
\item In both cases, the discriminant of $f$ is equal to $e^2v^2$ and the discriminant of number field $K$ is equal to $e^2$.
\item Conversely, if $e$ is equal to 9 times the product of $t-1$ distinct primes congruent to 1 modulo 3, (resp. is equal to the product of $t$ distinct primes congruent to 1 modulo 3), then there exists up to isomorphism exactly $2^{t-1}$ cyclic cubic fields of discriminant $e^2$ defined by the polynomials $f(x)$ given in (1) (resp. (2)).
\end{enumerate}
\end{theorem}

For this case, the integral basis is showed as follows.
\begin{theorem}\label{thm:ccpolycon}
With the notation in Theorem \ref{thm:ccpoly}, the conjugates of $\theta$ are given by the formulas:
\begin{enumerate}
\item If 3 is ramified in $K$,(i.e. in case (1)) then $$\sigma^{\pm1}(\theta)=\mp\frac{2e}{9v}+\frac{-3v\mp u}{6v}\theta\pm\frac{\theta^2}{v};$$
\item If 3 is unramified in $K$,(i.e. in case (2)) then $$\sigma^{\pm1}(\theta)=\frac{9v\pm(u+2-4e)}{18v}+\frac{-3v\mp (u+4)}{6v}\theta\pm\frac{\theta^2}{v}$$
In all cases, $(1,\theta,\sigma(\theta))$ is an integral basis of $K$. 
\end{enumerate}
\end{theorem}

\section{Prime Decomposition}\label{sec:ccf-decom}
Without loss of generality, we use the symbol in theorem \ref{thm:ccpoly} for simplicity. The situation of the decomposition of prime number in cyclic cubic field is quite easy, from the transitivity properties of Galois group, there are only three cases for decomposition, inert, totally ramified, splits completely. 

From the fundamental theorem of ramification, i.e. theorem \ref{thm:ramification}, it follows that if $p|e$, i.e. $p$ is one of the prime number (including 3) lies in $e$'s factorization expression, then $p$ is totally ramified. If $p\nmid e$ , then $p$ is unramified, even from theorem \ref{thm:unramified}, we can not determine the inert or splits completely. However if $p\nmid v$, then we can use the theorem \ref{thm:kummer} to determine the prime decomposition. In fact, the result is similar to quadratic field but change the Kronecker symbol to a cubic residue symbol, i.e. consider the congruence $x^3\equiv e(\operatorname{mod} p)$ is solvable or not.(or equivalently $p^{(e-1)/3}\equiv 1(\operatorname{mod} p)$).

If $p\mid v$, then $f$ has at least a double root modulo $p$. If $f$ has a double root, but not a triple root, then $f$ also has a simple root which corresponds to a prime ideal of degree 1. In this case $pO_K$ is the product of three ideals of degree $1$, i.e. splits completely. Finally, if $f$ has a triple root modulo $p$, we must apply other techniques, see Cohen's book 6.2.5 \citep{cohen1993course}. 


\section{Units and Class Number}
From Dirichlet Unit Theorem, since $K$ is a real field, hence $\mu(K)=\{\pm1\}$ and we have $U(K)=\{\pm1\}\times{\mathbb{Z}^2}$. A unit $\tau$ of $K$ is called the fundamental unit of $K$ if and only if $\{-1,\tau,\sigma(\tau)\}$ generate the group of units of $K$.
From the property of units, we have $\operatorname{N}(\tau)=\pm1$. The only fundamental units are $\pm\tau^{\pm1},\pm(\sigma(\tau))^{\pm1},\pm(\sigma^2(\tau))^{\pm1}$. The fundamental unit $\tau$ is uniquely determined except for taking conjugates and inverses.

This fundamental system of units can be calculated by means of generalized continued fraction algorithms by Voronoi \citep{voronoi1896generalization}, which have been interpreted geometrically by Delone and Faddeev \citep{delone1964theory}.

Harvey Cohn and Saul Gorn first give the units of 45 cyclic cubic fields of discriminants, where $e$ is the prime congruent to 1 modulo 3 between 7 to 499. 

Then, a table of class numbers and units in cyclic cubic fields with $e<4000$\footnote{this $e$ is the same to the one in Theorem \ref{thm:ccpoly}} has been given by Marie-Nicole Gras \citep{gras1975methodes}, after that Veikko Ennola and Reino Turunen \citep{ennola1985cyclic} have constructed an extended table for $e<16000$. 


\section{Conductor of Cyclic Cubic Field}
From the definition of conductor and some results in chapter \ref{chap:chap-two}, we can get the conductor of a cyclic cubic field $f_3$ is of the form \citep{maki1980determination} 

$$f_3=\left\{\begin{array}{cc}q_0q_1\cdots q_n& if 3\nmid f_3\\ 9q_1\cdots q_n&if 3\mid f_3\end{array}\right.$$
where $q_i$ are pairwise distinct rational primes satisfying $q_i\equiv1(\operatorname{mod} 3)$ for $i=1,2,\dots,n$. And the discriminant of $K_3$ is $d_3=f_3^2$.

\section{Examples}
In this section, we would propose some examples on prime decomposition and computing class number.

\subsection{An Example for Prime Decomposition}
\begin{example}
Consider $K=\mathbb{Q}(\theta)$, where $\theta$ is a root of $f(x)=x^3-93x-124$, it's easy to compute the discriminant of it is $e^2=31^2=961$, and $v=2$. Hence, we have $p=31$ is ramified totally in $O_K$, while the other special prime is $p=2$, since $2\mid v$. Consider $f(x)\equiv x(1+x)^2 \mod 2$, hence $p=2$ is splits completely in $O_K$. Other prime numbers can be determined by the Kummer's theorem \ref{thm:kummer}. Note that for $p=2$, this decomposition in $\mathbb{F}_2$ has no meaning in Dedekind's criterion \ref{thm:Dedekindcri}. 
\end{example}

\subsection{Some Examples for computing class group}
First of all, we consider the class group of the the number field $K=\mathbb{\theta}$, where $\theta$ is a root of $f(x)=x^3-3x+1$, since the discriminant of the polynomial is $81=9^2$, hence $K$ is a cyclic cubic field. From Theorem \ref{thm:cbfintegralnor}, we get the integral basis of the field is $(1,\theta,\sigma(\theta))=(1,\theta,\theta^2-2)=(1,\theta,\theta^2)$, i.e. it has a power integral basis, $O_K=\mathbb{Z}[\theta]$. The discriminant of the field is $d(K)=(9)^2=81$, then from the definition of Minkowski's bound, we have $C_K=\frac{2}{9}|81|^{1/2}=2$, hence $Cl(K)=\langle[\wp]|\wp|2\rangle$.
Now we consider the decomposition of 2, for $p=2$, $f(x)\equiv x^3+x+1 (\operatorname{mod} 2)$ is an irreducible polynomial, i.e. $2$ is inert in $K$. Hence $Cl(K)=\{1\}$, and $h(K)=1$.

Then we see another example, the number field $K=\mathbb{Q}(\theta)$ with $\theta$ is a root of $f(x)=x^3-x^2-4x-1$. This is a standard form of cyclic cubic field with the secdond case of Theorem \ref{thm:ccpoly}, we yield that $e=13,u=5,v=1$ for this case. Hence, the discriminant of the field is $13^2=169$, with $v=[O_K:\mathbb{Z}[\theta]]=1$, hence for any case, we can use theorem \ref{thm:kummer}, since $K$ is monogenic with integral basis $(1,\theta,\theta^2)$. From the definition of Minkowski's bound, we have $C_K=\frac{2}{9}|169|^{1/2}=2.889<3$, hence just consider $p=2$. Since $f(x)\equiv x^3+x^2+1 (\operatorname{mod} 2)$ is irreducible, i.e. we have $2$ is inert in $K$. Hence $Cl(K)=\{1\}$, and $h(K)=1$.

\section{Some Special Cyclic Cubic Fields}\label{sec:ssccf}
In this section, we'd like to share some cyclic cubic fields which have been researched by some mathematicians.
\subsection{Cubic Trinomials}
First, we would like to introduce F.C. Orvay's work \citep{orvay1991cyclic}. Some important results for the structure of cyclic cubic fields with cubic trinomials are listed as followings. For the sake of brevity, we just write down these lemma without proofs. These results can be found in F.C. Orvay's article.

\begin{lemma}
Let $K=\mathbb{Q}(\theta)$, $\operatorname{Irr}(\theta,\mathbb{Q})=x^3-px+p$ \footnote{Note that this is not the standard form of cyclic cubic field, however, it is isomorphisic to $f(x)=x^3-3px-(4p-27)p$.}, $p=3^{\delta}p_1\cdots p_r, p_i\equiv1(\operatorname{mod} 3)$ (distinct), $\delta\in\{0,2\}$, $4p-27\in\mathbb{Z}^2$, then the following hold:
\begin{enumerate}
\item $d(K)=p^2$, $K$ is monogenic\footnote{with power integral basis} with integral basis $\{1,\sigma,\sigma^2\}$.
\item $\{\sigma,\sigma'\}$ is a system of fundamental units of $K$, where $\sigma=(m+\theta_1)/3$, $\sigma'$ denote the conjugate of $\sigma$, $\theta_1=(4p-9\theta-6\theta^2)/\sqrt{4p-27}$ and $m=(\sqrt{4p-27}-3)/2$.
\end{enumerate}
\end{lemma}

This lemma is suitable for $p=9,13,19,37,63,79,97,117,139,163,\cdots$.

\begin{lemma}
Let $K=\mathbb{Q}(\theta)$, $\operatorname{Irr}(\theta,\mathbb{Q})=x^3-px+pq, p=p_1\cdots p_r, p_i\equiv1(\operatorname{mod} 3)$ (distinct),$q>2$, $4p-27q^2=1$, then the following hold:
\begin{enumerate}
\item $d(K)=p^2$, $K$ is monogenic with integral basis $\{1,\theta,\theta^2\}$.
\item $\{\mu,\mu'\}$ is a system of fundamental units of $K$, where $\mu=2+3\sigma+3\tau$, $\sigma=(-1+\theta_1)/3$,$\tau=(\sigma^2+((q+1)/2)\sigma)/q$, $\theta_1=4p-9q\theta-6\theta^2$ and $\mu'=-1-6\sigma+3\tau$.
\item $\operatorname{Irr}(\mu,\mathbb{Q})=x^3-3((1+9q)/2)x^2+((27q-3)/2)x+1$.
\item $\{1,\sigma,\tau\}$ is another integral basis.
\end{enumerate}
\end{lemma}

This lemma is suitable for $p=61,331,547,817,1141,\cdots$.

For the first lemma, on choosing $p$ more precisely , we have found a results as follows:
\begin{theorem}\label{thm:ccuformula}
For any integer $k$, set $p=k^2+k+7$. The polynomial  $X^3-p X+p$ is irreducible over $Q$ and has Galois group $A_3$.  
\begin{proof}
For any odd number $p$, $x^3-px+p\equiv x^3+x+1 (\operatorname{mod} 2)$, hence $x^3-px+p$ is irreducible over $\mathbb{Q}$. Its discriminant is $(-4)(-p)^3-27p^2=p^2(4p-27)$. To have a Galois group $A_3$, we need $4p-27\in\mathbb{Z}^2$. Writing $c^2=4p-27$, then we have $p=\frac{1}{4}(c^2+27)$. To make it integral we need $c$ odd, and write $c=2k+1$, then $$p=\frac{1}{4}(4k^2+4k+28)=k^2+k+7.$$ For any $k,k^2+k+7$ is odd so if we defined this expression to be $p$, then $x^3-px+p$ has Galois group $A_3$ over $\mathbb{Q}$.
\end{proof}
\end{theorem}

\subsection{Simplest Cubic Fields}
There is another special class of cyclic cubic fields called the simplest cubic fields first studied by Daniel Shanks \citep{shanks1974simplest}, Shanks computed the discriminant of the polynomial, fundamental units of the field, the regulator and some class number. A recent job for this type of cyclic cubic fields can be find in Lang's article \citep{lang2009properties}.

The simplest cubic field is defined by the following polynomial, 
\begin{equation}
f(x)=x^3-ax^2-(a+3)x-1,\label{eqn:simpcubic}
\end{equation} where $a\in\mathbb{Z}$. One can calculate the discriminant from the formula of the discriminant of polynomials (Refer to \ref{prop:disctri}), then $\operatorname{Disc}(f)=(a^2+3a+9)^2$.

D. Shanks only focuses on that $e:=a^2+3a+9$ is a prime, then from theorem \ref{thm:ccpoly}, there is only one cyclic cubic satisfies this polynomial equation up to isomorphism, and $e$ is also discriminant of the field.

What's more, one can verify that if $\theta$ is a root of equation \ref{eqn:simpcubic}, then $\theta'=1/(\theta+1)$ and $\theta''=1/(\theta'+1)$ are also a root of the equation \ref{eqn:simpcubic}.
And since $\theta(\theta^2-a\theta-a-3)=1$, i.e. $\theta$ is a unit, so is $\theta'$, then $1+\theta=-1/\theta'$ is also a unit. In fact, $(\theta,1+\theta)$ are independent fundamental
units, which can be verified by Godwin's criterion \citep{godwin1960determination}. More generally, we have following theorem which could be found in Lang's article \citep{lang2009properties}.
\begin{theorem}
If $e:=a^2+3a+9$ is square-free, then $(1,\theta,\theta^2)$ is an integral basis of $K$, and $(-1,\theta,\theta')$ generates the full group of units $O_K$.
\end{theorem}

A simple result about decomposing prime is that $2$ is inert in $K$, since $O_K=\mathbb{Z}[\theta]$ and $\bar{f}(x)\equiv x^3+x^2+1 (\operatorname{mod} 2)$ when $m$ is odd, while $\bar{f}(x)\equiv x^3+x+1 (\operatorname{mod} 2)$ when $m$ is even.

The explicit solution (a positive root) of equation \ref{eqn:simpcubic} is $$\theta=\frac{1}{2}(2\sqrt{e}\cos\phi+a),$$ where $$\phi=\frac{1}{3}\arctan\frac{\sqrt{27}}{2a+3}$$

As for the class number of them, D. Shanks \citep{shanks1974simplest} finally give a formula to compute them:
\begin{equation}
h=\frac{a^2+3a+9}{4\log^2 a}\left[1-\frac{3}{a\log a}+o(\frac{1}{a^2})\right]\prod_{p=2}^{\infty}f(p).
\end{equation}
where $f(p)$ is defined by
\begin{equation}
f(p)=\left\{\begin{array}{ll} 1& \text{for } p=e,\\ \left(\frac{q}{q-1}\right)^2 & \text{for } p^{(e-1)/3}\equiv 1(\operatorname{mod} p),\\
\frac{q^2}{q^2+q+1} & \text{otherwise}.
\end{array}\right.
\end{equation}

Shanks give a table of class number for $-1\leq a\leq 410$, where $e$ is a prime.
  \chapter{Cyclic Sextic Fields}
\label{chap:chap-six}
In this chapter, we will study cyclic sextic number fields. First of all, the result of discriminant of a cyclic sextic field could be found in M\"{a}ki's book \citep{maki1980determination}, while we would recover the result using the conductor-discriminant formula \ref{thm:conddisc}. Then, from a necessary theorem given by M\"{a}ki, together with the result of discriminant, we give the integral basis of a cyclic sextic number field. After that, with the help of the fundamental ramification theorem \ref{thm:ramification}, Stickelberger's theorem \ref{thm:unramified} on unramified primes and decomposition results for cyclic cubic field and quadratic field, we obtain the prime decomposition for cyclic sextic number fields. As for the unit groups and class numbers for real cyclic sextic fields, M\"{a}ki has solved those problem, and give a table on them. For complex case, it's a CM-field, hence the class number and the unit group could be reduced into its cyclic cubic field. But unfortunately, we haven't got the final precise results of them.

Let $K_6$ be a cyclic sextic number field over $\mathbb{Q}$ with Galois group $G=C_6$, then from Lemma \ref{lem:signaturegal}, we have that the signature of $K_6$ can only be $(6,0)$ or $(0,3)$ i.e. $K_6$ is totally real or totally complex field. It's easy to see that $G$ has exactly two nontrivial subgroups, namely those are of order 2 and order 3, thus the field $K_6$ has exactly two nontrivial subfields: a (real) cyclic cubic field $K_3$ and a quadratic field $K_2$.

Consider $K_2=\mathbb{Q}(\sqrt{m})$ where $m$ is a square-free integer. $K_3=\mathbb{Q}(\theta)$. Hence, $K_6=\mathbb{Q}(\theta,\sqrt{m})$. Let an odd integer $s$ be such that the automorphism $\sigma$ induced by the mapping $\zeta_{f_6}\rightarrow \zeta_{f_6}^s$, where $f_6$ is the conductor of $K_6$, satisfies the conditions:
$$\sigma(\theta)=\theta',\sigma(\theta')=\theta'',\sigma(\sqrt{m})=-\sqrt{m}$$
where the $\theta'$ and $\theta''$ denote the conjugates of $\theta$ in the cyclic cubic subfield, which are defined in theorem \ref{thm:ccpolycon}. What's more, we use the following notations $\gamma^{(i)}=\sigma^{i}(\gamma), i\in\mathbb{Z}$. For simplicity, we will continue to use these notations for the following content.


\section{Discriminant and Integral Basis}
From the Kronecker–Weber theorem, if $K_6$ is a subfield of a cyclotomic field $\mathbb{Q}(\zeta_k)$ then also $Q(\zeta_{f_2})$ and $Q(\zeta_{f_3})$ are contained in $Q(\zeta_{k})$. Hence, we have
$$f_6=\operatorname{lcm}(f_2,f_3)$$
As we all known, the conductor $f_2$ of real quadratic field $\mathbb{Q}(\sqrt{m})$ is equal to its fundamental discriminant $D$, and the conductor $f_3=e^2$ as we mentioned in \ref{thm:ccpoly}.
The characters of $K_6$ \citep{maki1980determination} are the principal character $1$, the quadratic character $\chi_2$ of $K_2$, the generating characters $\chi_3$ and $\bar{\chi}_3$ of $K_3$ and the generating characters $\chi_6=\chi_2\chi_3$ and $\bar{\chi}_6=\chi_2\bar{\chi}_3$ of $K_6$. The conductor of the character $\chi_n$ and $\bar{\chi}_n$ is $f_n$ (since they service for same cyclic galois group). Hence we have following results \citep{hasse1952uber}:
\begin{theorem}\label{thm:sexdisc}
The discriminant $d(K_6)$ of the real field $K_6$ is $$d(K_6)=f_6^2f_3^2f_2.$$
\end{theorem}
The proof is easy, as we mentioned above, we have found all characters of $K_6$, then by the conductor-discriminant formula \ref{thm:conddisc}, we immediately get $$d(K_6)=f_6^2f_3^2f_2.$$ 
What's more, for complex cyclic sextic fields, we have similar results:
\begin{theorem}\label{thm:sexdisccom}
The discriminant $d(K_6)$ of the complex field $K_6$ is $$d(K_6)=f_6^2f_3^2f_2.$$
\end{theorem}

The next theorem \citep{maki1980determination} gives a necessary condition for a number $\alpha\in K_6$ to belong to $O_{K_6}$. For simplicity, let us denote $\operatorname{gcd}(f_2,f_3)$ to be $f_*$.
\begin{lemma} \label{integersex}
If $\alpha\in O_{K_6}$, then $\alpha$ is of the form $$\alpha=\frac{1}{2}(x_0+x_1\theta+x_2\theta')+\frac{1}{2f_*}(y_0+y_1\theta+y_2\theta')\sqrt{m}$$
where $\frac{1}{2}x_i+\frac{1}{2}y_i\sqrt{m}\in O_{K_2},(i=0,1,2)$.
\begin{proof}
Let $\alpha=a_0+a_1\theta+a_2\theta'+(b_0+b_1\theta+b_2\theta')\sqrt{m}$, where $a_i,b_i\in\mathbb{Q}(i=0,1,2)$ be a number of $O_{K_6}$. Then from the definition of $\sigma$, we have $$\alpha+\alpha^{(3)}=2(a_0+a_1\theta+a_2\theta')\in O_{K_3}.$$ Recall the theorem \ref{thm:ccpoly}, we have $(1,\theta,\theta')$ is an integral basis of $K_3$. So we have $a_i=x_1/2$, where $x_i\in\mathbb{Z}(i=0,1,2)$. Also we have $$\sqrt{m}(\alpha-\alpha^{(3)})=2m(b_0+b_1\theta+b_2\theta')$$ is an algebraic integer. Hence $b_i=z_i/(2m)$ where $z_i\in\mathbb{Z}(i=0,1,2)$. Let $\lambda_i=\frac{1}{2}x_i+\frac{1}{2m}z_i\sqrt{m}$. Now we have the equations:
\begin{eqnarray}
\alpha&=&\lambda_0+\lambda_1\theta+\lambda_2\theta' \\
\alpha^{(4)}&=&\lambda_0+\lambda_1\theta'+\lambda_2\theta'' \\
\alpha''&=&\lambda_0+\lambda_1\theta''+\lambda_2\theta 
\end{eqnarray}
the determinant of the system of linear equation is $\pm f_3$ ($-f_3$ actually, but no need to determine the sign), since this can be viewed as the determination of integral basis and its conjugates. From the definition of discriminant, we have the discriminant is a square root of the discriminant, namely $\pm f_3$. By any case, this follows that the number $f_3\lambda_i (i=0,1,2)$ are algebraic integers. 

Then, the numbers $f_3z_i/m (i=0,1,2)$ are rational integers. Since $f_3$ is odd, $\operatorname{gcd}(m,f_3)=f_*$. Hence $z_i/m=y_i/f_*$ where $y_i\in\mathbb{Z}(i=0,1,2)$, and $\frac{1}{2}x_i+\frac{1}{2}y_i\sqrt{m}\in O_{K_2}$, because $f_3$ is odd and $f_3\lambda_i$ is integral.
\end{proof}
\end{lemma}
Note that if $f_*=\operatorname{gcd}(m,f_3)=1$, it follows that $O_{K_6}=O_{K_2}O_{K_3}$, which coincides the result which can be found in \citep{Xianke2006ANT,cohen1993course} etc. 

Next we will using above lemma to determine the integral basis of the cyclic sextic field. 
\begin{theorem}\label{thm:intbasis_sex}
The integral basis of the cyclic sextic field $K_6=\mathbb{Q}(\theta,\sqrt{m})$($m$ is a square-free integer. ) is of form $$(1,\theta,\theta',\eta,\eta\theta,\eta\theta'),$$ where $\eta=\frac{Df_*+\sqrt{D}}{2f_*}$, $f_*=\operatorname{gcd}(f_2,f_3)$ and $D=f_2$ is the (fundamental) discriminant of the quadratic subfield.
\begin{proof}
From lemma \ref{integersex}, it's easy to show that each element in the basis is of course integral in $K_6$. So we need to calculate the discriminant of these elements. Let $A$ denote the matrix $$\left(\begin{array}{lll} 1&\theta&\theta'\\ 1&\theta'&\theta'' \\1&\theta''&\theta\end{array}\right),$$
and $$B=\left(\begin{array}{lll} \eta&\eta\theta&\eta\theta'\\ \eta&\eta\theta'&\eta\theta'' \\ \eta&\eta\theta''&\eta\theta\end{array}\right),C=\left(\begin{array}{lll} \bar{\eta}&\bar{\eta}\theta&\bar{\eta}\theta'\\ \bar{\eta}&\bar{\eta}\theta'&\bar{\eta}\theta'' \\ \bar{\eta}&\bar{\eta}\theta''&\bar{\eta}\theta\end{array}\right)$$
where $\bar{\eta}=\frac{Df_*-\sqrt{D}}{2f_*}$, then the discriminant of the elements is following:
\begin{eqnarray*}
&&\operatorname{Disc}(1,\theta,\theta',\eta,\eta\theta,\eta\theta')\\
&=&\left(\det\left(\begin{array}{ll} A & B \\ 0&C-B \end{array}\right)\right)^2\\
&=&(\det{(A)}\cdot\det{(C-B)})^2\\
&=&\left(-f_3\cdot((\sqrt{D})^3\cdot(-f_3)/f_*)\right)^2\\
&=&\frac{f_3^4D^3}{f_*^2}\\
&=&f_3^2f_2\frac{f_3^2f_2^2}{f_*^2}\\
&=&f_3^2f_2f_6^2
\end{eqnarray*}
i.e. we have the basis's discriminant is equal to the discriminant of the field, hence $$(1,\theta,\theta',\eta,\eta\theta,\eta\theta')$$ is an integral basis.
\end{proof}
\end{theorem}

\section{Prime Decomposition}\label{sec:primdsex}
For the prime decomposition in cyclic sextic field, we should consider the situation of its two subfields, which are solved in Chapter \ref{chap:chap-four} and \ref{chap:chap-five}. First of all, let us consider the ramified primes in the cyclic sextic field. As we known in previous theorem, the discriminant of the cyclic sextic field is $d(K_6)=f_2f_3^2f_6^2$, from the fundamental ramification theorem \ref{thm:ramification}, we should consider the prime numbers and discriminant. Since $f_6=\operatorname{lcm}(f_2,f_3)$, hence if $p\mid d(K_6)$, then $p\mid f_2$ or $p\mid f_3$. More precisely, form the chain rules for ramification index and decomposition number, we can immediately get the following results.
\begin{theorem}\label{thm:csf-ramified}
Let $K_6=\mathbb{Q}(\theta,\sqrt{m})$ be the cyclic sextic field, suppose $f_3=3^{\delta}p_1p_2\cdots p_t$, and $f_2=D=2^{\delta}q_1q_2\cdots q_s$ where $\delta=\{0,2\}$,
$p_i\equiv 1 (\operatorname{mod} 3)$ and $q_i\equiv 1 (\operatorname{mod} 2)$; $s,t$ are rational integers. What's more, without loss of generality, suppose $p_i=q_i$ for $0<i\leq t_0\leq \min{\{s,t\}}$, Then we have 
\begin{enumerate}
\item if $p=p_i=q_i$, where $0<i\leq t_0\leq \min{\{s,t\}}$, then $p$ is totally ramified, i.e. $pO_{K_6}=\wp^6$
\item if $p=p_i$, where $i>t_0$, then $p$ is ramified in the subfield $K(\theta)$, and we have $pO_{K_6}=\wp_1^3\wp_2^3$ if $\left(\frac{f_2}{p}\right)=1$ or $pO_{K_6}=\wp^3$ if $\left(\frac{f_2}{p}\right)=-1$.
\item if $p=q_i$, where $i>t_0$, then $p$ is ramified in the subfield $K(\sqrt{m})$, and we have $pO_{K_6}=\wp_1^2\wp_2^2\wp_3^2$ or $pO_{K_6}=\wp^2$. To determine which case is suitable for $p$, we should used the results in section \ref{sec:ccf-decom} in chapter \ref{chap:chap-five}.
\end{enumerate}
\end{theorem}

As for the results for unramified cases, suppose $p$ is an unramified prime in $O_{K_6}$, i.e. $p\nmid f_2$ and $p\nmid f_3$, then from the theorem \ref{thm:unramified}, we have 
\begin{theorem}\label{thm:csf-unramified}
Let $K_6=\mathbb{Q}(\theta,\sqrt{m})$ be the cyclic sextic field, $p$ be an unramified prime in $O_{K_6}$, then 
\begin{enumerate}
\item if $\left(\frac{d(K_6)}{p}\right)=1$, then $pO_{K_6}=\prod_{i=1}^6\wp_i$ (splits completely), or $pO_{K_6}=\wp_1\wp_2$. 
\item if $\left(\frac{d(K_6)}{p}\right)=-1$, then $pO_{K_6}=p$ (inert) or $pO_{K_6}=\wp_1\wp_2\wp_3$.
\end{enumerate}
\end{theorem}
To determined the exact cases for above unramified theorem, we should refer to chapter \ref{chap:chap-five}.
 
\section{Unit and Class Number of Real Cyclic Sextic Field}
For real cyclic sextic field, we  have the signature of $K_6$ can only be $(6,0)$, then according to Dirichlet's theorem \ref{thm:Dirichlet} on units in $K_6$ there are 5 fundamental units, which together with $-1$ generate the multiplicative group $U_6$ of units of $K_6$. The unit group $U_6$ has the unit groups $U_2$ and $U_3$ of the subfields $K_2$ and $K_3$ as subgroups. 

As we known before, suppose $U_2$ is generated by $-1$ and the fundamental unit $\mu$, and $U_3$ is generated by $-1$, a fundamental unit $\tau$ and one of its conjugates $\tau'=\sigma(\tau)$. From Latimer's work \citep{latimer1934units}, we know that $K_6$ has a system of fundamental units containing $\mu,\tau,\tau'$. So there are three fundamental units are known, i.e. those belonging to the proper subfields, namely $\epsilon,\tau,\tau'$.  

To determine the other two units we should first get a so called cyclotomic unit which is calculable from a definite expression. This unit, together with its conjugates, and the units of the proper subfields generate a subgroup of finite index in the whole unit group, and it is in principle relatively easy to obtain the whole group from this subgroup.

First of all, we need a so-called relative units, a unit $\epsilon$ of $K_6$ for which $N_{6/3}(\epsilon)=\pm1$ and $N_{6/2}(\epsilon)=\pm1$ is called a \textbf{relative units}. Let $U_R$ denoted the group of relative units, i.e. we have $$U_R=\{\epsilon\in U_6|N_{6/3}(\epsilon)=\pm1,N_{6/2}(\epsilon)=\pm1\}$$

Note that if $\epsilon\in U_R$, then $N_{6/1}(\epsilon)=N_{2/1}(N_{6/2}(\epsilon))=N_{2/1}(\pm1)=1$. On the other hand, $1=N_{6/1}(\epsilon)=N_{3/1}(N_{6/3}(\epsilon))=(N_{6/3}(\epsilon))^3$ so that for $\epsilon\in U_R$, we have $N_{6/3}(\epsilon)=1$

What's more, M\"{a}ki has proved that in $K_6$, there exists a generating relative unit $\xi_R$ such that $$U_R=\{\pm\xi_R^k\xi_R'^l|k,l\in\mathbb{Z}\}.$$ On the contrary, every relative unit in $K_6$ has a unique representation in this form. M\"{a}ki also showed that how to calculate $\xi_R$ and then establish a solution of the unit.

As for the class number, using the cyclotomic unit, M\"{a}ki has prove a theorem that $h_6=h_2h_3h_R$, where $h_2,h_3$ are class number of the subfield $K_2$ and $K_3$, and the $h_R$ is so-called relative class number of $K_6$. These numbers can be calculated by the cyclotomic units together with fundamental units.

M\"{a}ki lists a huge number with conductor $f_6\leq 2021$ of cyclic sextic field with class number and unit group in her book. Then, together with Ennola and Turunen, she extend this table for $f_6<4000$.


\section{Unit and Class Number of Complex Cyclic Sextic Fields}\label{sec:unitccsf}
For complex situation, then the signature of $K_6$ can only be $(0,3)$, then according to Dirichlet's theorem \ref{thm:Dirichlet} on units in $K_6$ there are 2 fundamental
units, which together with $\mu(K_6)$ generate the multiplicative group $U_6$ of units of $K_6$.

To start discussing the unit and class number of complex cyclic sextic fields,  we first introduce the CM-field. Then idea of CM-field is the extension of complex quadratic field and cyclotomic field.  A CM-field $K$ is a totally imaginary extension of a totally real number field $k^+$, i.e. $$K=K^+(\sqrt{a}),$$ where $K^+$ is totally real, and $a\in K^+$ has negative conjugates.

For example, $K=\mathbb{Q}(\zeta_m)$ is a CM-field, since $K=K^+(\sqrt{a})$, where $K^+=\mathbb{Q}(\zeta+\zeta^{-1})$, and $a=\zeta^2+\zeta^{-2}-2$.

The relationship between the largest totally real subfield and the CM-field are very useful:
\begin{theorem} 
Let $\mathbb{Q}\subset F\subset K$ be nontrivial extensions of number fields. Then $K$ is a CM-field, with $F$ its totally real subfield, if and only if $U_K/U_F$ is finite.
\end{theorem}

Another useful theorem \citep{Xianke2006ANT} for class number and unit group is showed as follows:
\begin{theorem}\label{thm:unitcomplexsec}
Let $K$ be a CM-field, $K^+$ its largest totally real subfield, $h$ and $h^+$ be the class number of them respectively, $U$ and $U^+$ be the unit group of $K$ and $K^+$, then: 
\begin{enumerate}
\item $h^-=h/h^+$ is so-called the relative class number, which is a rational integer.
\item $Q:=[U:\mu(K)U^+]=1\text{ or }2$
\end{enumerate} 
\end{theorem}

More precisely, S. Louboutin has given the class number for complex cyclic sextic field  with $f_6\leq 220000$.
  \chapter{Examples for Cyclic Sextic Fields}
\label{chap:chap-seven}
In this chapter, we will give some examples for cyclic sextic field, using our results or methods given by chapter \ref{chap:chap-six} to a series of  problems: discriminant, integral basis, decomposition of prime, unit group and class number etc. For the first example in cyclotomic field, there exists a complete theory for that. But as a cyclic sextic number field, using the methods given by chapter \ref{chap:chap-six}, we also got the structures of it. More precisely, we discover two subfields of it, and then calculate the discriminant which is equal to the result in cyclotomic fields' theorem. Then we also give the prime decomposition of it based on our theorems in section \ref{sec:primdsex}. Also, we calculate the class number using the general theory. To solve the second field in section \ref{sec:rcsfexam}, we use almost all preliminaries we mentioned before. We first determine the Galois group of it, then we calculate the polynomial discriminant of it, based on the discriminant formula, then we find two subfields of it. After that, we also find its integral basis and get the prime decompositions. Also, we have tried to calculate the class number of it using general theory, and get the same result as M\"{a}ki did. The third section is another example given by A Bremner and B Spearman\citep{bremner2010cyclic} with sextic trinomial, but unfortunately, we haven't got the final results.

\section{7-th Cyclotomic Field}
\subsection{Cyclotomic Fields}
Let $\zeta_n$ denote a fixed primitive $n^{\text{th}}$ root of unity, and let $\mathbb{Q}(\zeta_n)$ be the number field generated by all the $n^{\text{th}}$ root of unity. The field $\mathbb{Q}(\zeta_n)$ is called the $n^{\text{th}}$ \textbf{cyclotomic field}. A following result is very import theorem.
\begin{theorem}
Let $\phi(n)$ denote the (Euler) totient of $n$, then $\mathbb{Q}(\zeta_n)$ is an Abelian extension of $\mathbb{Q}$ of degree $\phi(n)$. More precisely, there is an isomorphism:
\begin{eqnarray*}
(\mathbb{Z}/n)^{\times}&\rightarrow & \operatorname{Gal}(\mathbb{Q}(\zeta_n)/\mathbb{Q})\\
\bar{a}&\mapsto & \sigma_a,
\end{eqnarray*}
where $\sigma_a(\zeta_n)=\zeta_n^a$
\end{theorem}
Since a sub-extension of an Abelian extension is also Abelian, cyclotomic fields and their subfields already give us an abundant supply of Abelian extensions of $\mathbb{Q}$.
More formally, we have the 
\begin{theorem}[Kronecker-Weber]\label{thm:kronecker}
Every finite abelian extension of $\mathbb{Q}$ is contained in a cyclotomic field.
\end{theorem}

\subsection{7-th Cyclotomic Field}
Now let us consider the simplest cyclic sextic field, a 7-th cyclotomic field. As we all know, $K_6=\mathbb{Q}(\zeta_7)$ is generated by $\zeta_7$, which is a root of unity, and its minimal polynomial is $$f(x)=x^6+x^5+x^4+x^3+x^2+x+1.$$

First of all, we would know its structure. As we known in Section \ref{sec:unitccsf}, this field is a CM-field, i.e. a totally imaginary extension of a totally complex cyclic cubic field. Write as before, we have $K_6=\mathbb{Q}(\sqrt{m},\theta)$, where $m$ is a negative squarefree integer. Now we need to find the minimal polynomials for $\sqrt{m}$ and $\theta$.

Note that $\zeta_7^i$ is complex conjugate to $\zeta_7^{-i}$, so let $\xi=\zeta_7+\frac{1}{\zeta_7}$, then $\xi$ is a real number\footnote{One can find that $\xi=2\cos\frac{2\pi}{7}$.}. Since $\zeta_7$ is a root of the minimal polynomial, then we have $$\zeta_7+\frac{1}{\zeta_7}+\zeta_7^2+\frac{1}{\zeta_7^2}+\zeta_7^3+\frac{1}{\zeta_7^3}+1=0,$$ Substitute $\xi$ into the equation, we have \begin{equation}\label{eqn:xi}
g(\xi)=\xi^3+2\xi^2-2\xi-1=0
\end{equation}
Let $x=\xi-\frac{2}{3}$, we can change it into the standard form, i.e. $$x^3-x^2-2x+\frac{13}{27},$$
where we can get $e=7,u=-1,v=1$, hence, the cyclic cubic subfield defined by $\xi$, where $\xi$ is a solution of equation \ref{eqn:xi}. i.e. $K^+=K_3=\mathbb{Q}(\xi)$.

From theorem \ref{thm:ccpoly}, we got the discriminant of $K_3$ is $49=f_3^2$, and since $v=1$, we got that $(1,\xi,\xi^2)$ is a power integral basis of $K_3$.

On the other hand, note that the automorphism $\sigma_2:\zeta_7\rightarrow\zeta_7^2$ generates the subgroup of order 3. Thus consider $\omega=\zeta_7+\zeta_7^2+\zeta_7^4$.
From the properties of root of unity, we have following two equations: $$(\zeta_7+\zeta_7^2+ \zeta_7^4)+(\zeta_7^3+\zeta_7^6+\zeta_7^{12})=-1,$$
$$(\zeta_7+\zeta_7^2+ \zeta_7^4)(\zeta_7^3+\zeta_7^6+\zeta_7^{12})=3-1=2$$

Then $\omega$ satisfies a quadratic equation: $$h(x)=x^2+x+2=0,$$ hence $\mathbb{Q}(\omega)$ is a quadratic subfield, and $K_2=\mathbb{Q}(\sqrt{-7})$. Then the discriminant of $K_2$ is $-7=f_2$. From theorem \ref{thm:sexdisccom}, we have the discriminant of $K_6$ is $d(K_6)=(-7)\times(7)^2\times(7)^2=-7^5$.

A famous result for discriminant of $\mathbb{Q}(\zeta_p)$, where $p$ is a prime, is that $\operatorname{Disc}(\mathbb{Q}(\zeta_p))=(-1)^{(p-1)/2}p^{p-2}$. In this field, we have $\operatorname{Disc}(\mathbb{Q}(\zeta_7))=(-1)^{3}7^{7-2}=-7^5$, which coincides our result.

To sum up, we got that $K_6=\mathbb{Q}(\xi,\sqrt{-7})$. As for the prime decomposition, the discriminant has only one prime divisor, namely $7$, hence only $7$ is ramified in $K_6$. what's more, since $7$ is common factor of $f_2$ and $f_3$, hence it is totally ramified. In fact, since $K_6$ is monogenic, we have $(x+6)^6 (\operatorname{mod} 7)$. For other primes, we just need to consider the decomposition in subfields, the using the result in section \ref{sec:primdsex}. For example, consider $p=37$, note that $$\left(\frac{-7}{37}\right)=(-1)^{(37-1)/2}\left(\frac{7}{37}\right)=(-1)^{(7-1)(37-1)/4}\left(\frac{37}{7}\right)=-\left(\frac{2}{7}\right)=1$$ Hence $37$ is ramified in $K_2$. On the other hand, similarly, consider we just need to verify that $49^{(37-1)/3}\equiv 1 (\operatorname{mod }37)$ or not. However, $49^{(37-1)/3}\equiv 12^{12} \equiv (-4)^6\equiv (-10)^2\equiv 26\not\equiv 1 (\operatorname{mod} 37)$, hence $37$ is inert in $K_2$. Hence, $37=\wp_1\wp_2$.

As for the unit group of $K_6$, we first refer a theorem  \citep{Xianke2006ANT} which is a corollary of theorem \ref{thm:unitcomplexsec} as follows:
\begin{theorem}
Let $m=p^s$, where $p$ is an odd prime, $s\in\mathbb{N^*}$, then $K=\mathbb{Q}(\zeta_m)$ has the same system of fundamental units to $K^+=\mathbb{Q}(\zeta_m+\zeta_m^{-1})$.  
\end{theorem}

So from Dirichrlet Theorem \ref{thm:Dirichlet}, $U(K_6)=\mu(K_6)\times \mathbb{Z}^2$, where $\mu(K_6)=\langle\zeta_7\rangle$ is the root of unity of $K$ and the fundamental units are the same to $K_3$. The fundamental units in $K_3$ are $$(-1+\xi+\xi^2,2-\xi^2)=(1+\zeta_7+\zeta_7^2+\zeta_7^{-1}+\zeta_7^{-2},-\zeta_7^{-2}-\zeta_7^{2}).$$

As for the class number, firstly, we compute the Minkowski's bound, and we get $$C(K_6)=\left(\frac{4}{\pi}\right)^3\frac{6!}{6^6}\sqrt{|-7^5|}=4.13$$
Hence $Cl(K_6)=\langle[\wp]|\wp|2\text{ or }3\rangle$

now we consider the decomposition of $2$ and $3$. since $\left(\frac{-7}{2}\right)=1$, hence it splits in $K_2$. On the other hand, $x^3+x^2-2x-1\equiv x^3+x^2+1 (\operatorname{mod} 2)$, i.e. it's inert in $K_3$, hence $2=\wp_1\wp_2$\footnote{Here we haven't use cubic reciprocity, since there is no definition for $p=2$(in fact it only define in $p=3k+1$, like quadratic reciprocity defined in $p=2k+1$).}. In fact, $\wp_1=(2,1+\xi+\xi^3),\wp_2=(2,1+\xi^2+\xi^3)$, since $1+\xi^2+\xi^3$ is conjugate to $1+\xi+\xi^3$, hence $[\wp_1]=[\wp_2]=1$.

For $p=3$, it's easy to see that $\left(\frac{-7}{3}\right)=-1$ and $x^3+x^2-2x-1$ is irreducible in $\mathbb{F}_3$, hence $p=3$ is inert in $K_6$. To sum up, we have the class number is $1$.


\section{A Real Cyclic Sextic Field}\label{sec:rcsfexam}
Consider the sextic field $K_6$ generated by a root of $$f(x)=x^6-x^5-6x^4+6x^3+8x^2-8x+1.$$ First note that $f(x)$ is irreducible over $\mathbb{Q}$, since $f(x)\equiv x^6+x^5+1 (\operatorname{mod} 2)$. In order to verify the Galois group of the splitting field of $f$, we should first compute the resolvent polynomials. Let $p_1=x_1+x_2$, $p_2=x_1+x_2+x_3$, then use the numerical method, we have the approximate root of $f$ are: $$(r_1,r_2,r_3,r_4,r_5,r_6)=(-1.97766,-1.46610,0.14946,0.73068,1.65248,1.91115)$$
Then, \begin{eqnarray*}
R_{p_1,f}(y)&=&\prod_{p_{1i}\in orb(p)}(y-p_{1i}(r_1,\dots,r_6))\\
&=& x^{15}-5x^{14}-14x^{13}+98x^{12}+7x^{11}-567x^{10}+280x^9+1404x^8\\
&&-818x^7-1596x^6+735x^5+700x^4-203x^3-77x^2+11x+1\\
&=&(x^3-x^2-2 x+1)(x^6-2 x^5-10 x^4+6 x^3+30 x^2+17 x+1)\\
&&(x^6-2 x^5-10 x^4+27 x^3-12 x^2-4 x+1)
\end{eqnarray*} 

So we have $(3,6^2)$ cycle, from table \ref{tab:allorbits}, we have that the Galois group of $f$ is either $C_6$ or $D_6$. Then we calculate $R_{p_2,f}$,
\begin{eqnarray*}
R_{p_2,f}(y)&=&\prod_{p_{2i}\in orb(p)}(y-p_{2i}(r_1,\dots,r_6))\\
&=& x^{20} + 2 x^{19} - 106 x^{18} - 600 x^{17} - 593 x^{16} + 3252 x^{15} - 6530 x^{14}\\
&& -2589 x^{13} + 4875 x^{12} - 675 x^{11} - 1759 x^{10} + 3349 x^9 + 5376 x^8 + 1260 x^7 \\
&&+ 1188 x^6 + 865 x^5 + 316 x^4 + 77 x^3 + 23 x^2 + 11 x + 1
\\
&=&(x^2 - x + 1) (x^6 - 12 x^5 + 3 x^4 + 7 x^3 + 6 x^2 - 2 x + 1) \\
&&(x^6 + 8 x^5 + 25 x^4 + 2 x^3 + 5 x^2 + 2 x + 1) \\
&&(x^6 + 7 x^5 - 8 x^4 + 4 x^3 + 27 x^2 + 12 x + 1)
\end{eqnarray*}  
So we have $(2,6^3)$ cycle, from table \ref{tab:allorbits}, we have that the Galois group of $f$ is $C_6$, also we know $K_6$ is totally real.

The polynomial discriminant of $f$ could be given by formula \ref{prop:disctri}, in fact, $\operatorname{Disc}(f)=453789=3^3\times7^5$
Note that, $d(K_6)=f_2f_3^2f_6^2$ by theorem \ref{thm:sexdisc}, since $f_6=\operatorname{lcm}(f_2,f_3)$, hence we have $f_2^3|d(K_6)$ and $f_3^4|d(K_6)$. The relation between $d(K_6)$ and $\operatorname{Disc}(f)$ is showed in lemma \ref{lem:discpoly-field}, i.e. $\operatorname{Disc}(f)=a^2d(K_6)$, here we have $$3^3\times 7^5=a^2 f_2f_3^2f_6^2,$$ from the requirements we mentioned above, we have $$a^2f_3^4|3^3\times7^5,$$ this force $f_3=7$, and $a=1$ or $a=3$. Then $a^2f_2^3f_3^2|3^3\times7^5$, i.e. $$a^2f_2^3|3^3\times7^3.$$ If $a=3$, then $f_2=7$, and for this case $d(K_6)=7^5$, and we get $3^3d(K_6)=\operatorname{Disc}(f)$. An contradiction! Hence, $a=1$, and we could get $f_2=3\times7$.

To sum up, we have $f_2=21,f_3=7$. So we can immediately get the quadratic subfield is $\mathbb{Q}(\sqrt{21})$, with minimal polynomial $g(x)=x^2-x-5$, and integer ring $\mathbb{Z}\left[\frac{1+\sqrt{21}}{2}\right]$. \footnote{In fact, since $\phi(21)=\phi(3)\times\phi(7)=12$ and $6|12$, actually this cyclic sextic field is a subfield of the cyclotomic field $\mathbb{Q}(\zeta_21)$.}  As for the cyclic cubic subfield, since $e=f_3=7$, and $v=1$, hence we have the minimal polynomial of $\theta$, $h(x)=x^3-x^2-2x+\frac{13}{27}$ with power integral $(1,\theta,\theta^2)$. To sum up, $$K_6=\mathbb{Q}(\theta,\sqrt{21})$$

Since $d(K_6)=\operatorname{Disc}(f)$, hence $K_6$ has a power integral basis, wlog, let $r$ be a root of $f$, then $(1,r,r^2,\dots,r^5)$ is a integral basis. On the other hand, we have another integral basis given by theorem \ref{thm:intbasis_sex}.

The prime decomposition is quite easy, $7$ is totally ramified in $K_6$, since it is ramified in both two subfield. In fact, $f(x)\equiv (x+1)^6 (\operatorname{mod} 7)$. Hence $7O_{K_6}=\wp^6$, where $\wp=(7,r+1)$. $3$ is ramified in $K_2$, and inert in $K_3$, hence $2O_{K_6}=\wp'^2$. The other prime's situation could be solved through theorem \ref{thm:csf-unramified}.

As for the class number, we first compute the Minkowski's bound, $$C_{K_{6}}=(4/\pi)^{0}\frac{6!}{6^6}\sqrt{453789}=10.4,$$ now we should consider $p=2,3,5,7$. we can find that $2$ is inert in $O_{K_6}$. $7$ is totally ramified with $\wp=(7,r+1)$, while $N(r+1)=7$\footnote{calculate through $f(x-1)$}, $(r+1)\subset(7)$, hence $\wp=(r+1)$ is a principal ideal.

As for $p=3$, we have $f(x)\equiv(2+x+x^2+x^3)^2 (\operatorname{mod} 3)$, hence $\wp'=(3,2+r+r^2+r^3)$, $N(2+r+r^2+r^3)=3$ hence $\wp'=(2+r+r^2+r^3)$. Similar result for $p=5$. Finally, we get $h(K_6)=1$.

\section{Sextic Trinomials}
As expected, increasing the degree of a polynomial will increase its complexity. In the case of sextics, however, the number of possible Galois groups jumps up to sixteen. We can once again find a list of these groups in Cohen's book and we expect $S_6$ to be the most frequently occurring group. 

It is much more preferable to work with sextic trinomials rather than general sextics. Again, we can reduce the possible unique forms of these trinomials to only $x^6+ax+b,x^6+ax^2+b,x^6+ax^3+b$. Note that the last of these three forms can be simplified to a quadratic in $x^3$.

It has already been shown by A Bremner and B Spearman \citep{bremner2010cyclic} that up to scaling, there exists a single, unique sextic trinomial with Galois group isomorphic to $C_6$; which was given as
$$f(x)=x^6+133x+209$$

Now we focus on this example, i.e. $K_6$ is the splitting field of $f(x)$. First of all, since $f(x)\equiv x^6+x+1(\operatorname{mod} 2)$, hence $f(x)$ is irreducible over $\mathbb{Q}$.  Then, we can compute the discriminant of the polynomial, from formula \ref{for:nxpxq}, we have $$\operatorname{Disc}(f)=(-1)^{\frac{4\times5}{2}}\cdot 5^5\cdot 133^6+(-1)^{\frac{6\times5}{2}}\cdot 6^6\cdot 209^5=-19^5\times83^2\times277^2.$$

We have $d(K_6)=f_2f_3^2f_6^2$ by theorem \ref{thm:sexdisc}, since $f_6=\operatorname{lcm}(f_2,f_3)$, hence we have $f_2^3|d(K_6)$ and $f_3^4|d(K_6)$. The relation between $d(K_6)$ and $\operatorname{Disc}(f)$ is showed in lemma \ref{lem:discpoly-field}, i.e. $\operatorname{Disc}(f)=a^2d(K_6)$. Whatever, $f_3^4|d(K_6)$ force that $f_3=19$ and $-19|f_2$. There are several possible cases for $d(K_6)$:
\begin{equation*}
\left\{ \begin{array}{l}
f_3=19,f_2=-19;\\
f_3=19,f_2=-19\times 83;\\
f_3=19,f_2=-19\times 277;\\
f_3=19,f_2=-19\times 83\times 277.
\end{array} \right.
\end{equation*}
To recognize the case for the field generated by a root of $f$, we need another method. However, this example is still under solving.


  \chapter{Galois Extensions of Function Fields}
\label{chap:chap-eight}

L{\"u}roth's theorem is one of elementary results in the classical algebra which we can find in a textbook by Van Der Waerden \citep{van1967algebra}. L{\"u}roth \citep{luroth1875beweis} proved L{\"u}roth's theorem in case $K=\mathbb{C}$ in 1876. It was first proved for general fields $K$ by Steinitz \citep{steinitz1910algebraische} in 1910, by the above argument. An precisely elementary algebraic proof using field theory and Gauss's Lemma was given by G. Bergman as a series of exercises for students. The L{\"u}roth's theorem is showing as follows:
\begin{theorem}[L{\"u}roth's Theorem]
Let $K$ and $E$ be fields such that $K\subsetneq E \subset K(x)$, where $x$ is transcendental extension of $K$. Then $E=K(u)$ for $u\in K(x)$, and $K(x)$ is finite-dimensional over $E$.
\end{theorem}

In our work, we are focus on a special case in L{\"u}roth theorem, i.e. we add a condition on the field extension $K(x)/E$, namely we suppose that $K(x)/E$ is Galois. And for this case, we will first give a simple proof of the similar results. Of course, we won't use the result of L{\"u}roth's thorem. Then, we will give a explicit form of the invariant rational function due to the Galois group.

\section{Existence Theorem}
First, we'd like to give the following theorem: 
\begin{theorem}\label{thm:luroth-zwc}
Let $K$ and $E$ be fields such that $K\subsetneq E \subset K(x)$, where $x$ is transcendental extension of $K$ and $K(x)$ is Galois over $E$. Then $E=K(u)$ for $u\in K(x)$.
\end{theorem}

For simplicity, we start with a definition of height of a rational function.

\begin{definition}[height]
The \textbf{height} of a rational function $u(x)=f(x)/g(x)$ in a rational function field is defined to be $$\operatorname{ht(u)}=\max\{\deg(f(x)),\deg(g(x))\}$$ 
\end{definition}

Gauss's Lemma is necessary in our proof, which is either of two related statements about polynomials with integer coefficients. The first one is the primitivity statement and the second is the irreducible statement \citep{zieve2011mathematics}.

\begin{lemma}[Gauss's Lemma]
Let $R$ be a unique factorization domain.
\begin{enumerate}
\item The product of two primitive polynomials is primitive, what's more, for $f,g\in R[x]$, we have $C(f)C(g)= C(fg)$, where $C(f)$ is the great common divisor of the coefficients of $f$.
\item If $f\in R[x]\backslash R$ is irreducible in $R[x]$, then $f$ is irreducible in $Frac(R)[x]$.
\end{enumerate}
\end{lemma}

We give the following lemma \ref{lem:lurothfinitedim} which solve the finiteness of the extension in L{\"u}roth's theorem and our special case.

\begin{lemma}\label{lem:lurothfinitedim}
If $K$ is a field and $u(x)\in K(x)\backslash K$, then $[K(x):K(u)]=\operatorname{ht}(u)$.
\begin{proof}
Write $u(x)=a(x)/b(x)$, where $a,b\in K[x]$ are coprime. Then $x$ is a root of the polynomial $f:=a(T)-u(x)b(T)\in K(u)[T]$. Since this polynomial has degree one in $u$, then the coprimality of $a(T),b(T)$ implies that $f$ is irreducible in $(K[T])[u]=(K[u])[T]$. 
From Gauss's Lemma, we have that $f$ is irreducible in the fractional polynomial field $K(u)[T]$, i.e. $f$ is a constant multiple of the minimal polynomial of $x$ over $K(u)$. i.e. we have $[K(x):K(u)]=\deg_{t}(f)=\operatorname{ht}(u)$. 
\end{proof}
\end{lemma}

Since $K(x)/E$ is Galois, hence is a finite separable and normal extension. From the following so called primitive element theorem , we could know that $K(x)/E$ is finite simple extension.

\begin{lemma}[Primitive Element Theorem]
Let $E\supseteq F$ be a finite degree separable extension. Then $E=F(\alpha)$ for some $\alpha\in E$.
\end{lemma}  

\begin{proof}[Proof of Theorem \ref{thm:luroth-zwc}]
Since $K(x)/E$ is Galois, so is finite simple extension (simple algebraic extension).Since $K(x)=E(x)$, i.e. $K(x)$ can be viewed as a simple algebraic extension by adding $x$ into $E$. Then we may assume that $p(t)$ be the minimal polynomial of $x$ over $E$, $$p(t)=t^n+r_{n-1}t^{n-1}+\cdots+r_1t+r_0,(r_i\in E\subset K(x)).$$
Consequently, $n=\deg_t(p)=[K(x):E]$.
Let rewrite $r_i=a_i/b_i,(i=0,1,\dots,n-1)$ where $a_i,b_i\in K[x]$, and $a_i$ is coprime to $b_i$.  Then $p(t)$ could be multiplied by the l.c.m. of the $b_i$'s in order to get a primitive polynomial over the ring $K[x]$, namely $$q(t)=c_nt^n+c_{n-1}t^{n-1}+\cdots+c_0\in K[x][t].$$ Here, we get $\deg_t(q)=\deg_t(p)$

Note that (at least) one of the $r_i$'s, i.e. $u:=r_k\in E$, doesn't not belong to $K$, else $x$ would be algebraic over $K$. Let's consider the polynomial:
$$R(x,t)=a_k(t)b_k(x)-a_k(x)b_k(t)\in K[x,t]$$ 
%and $$S(t)=R(x,t)/b_k(x)=a_k(t)-r_k(x)b_k(t)\in E[t]$$ 
Since $R(x,x)=0$, $q(t)$ is the minimal polynomial of $x$, then $q(t)$ divides 
%$S(t)$ in $K(x)[t]$, then it also divides 
$R(x,t)$ in $K(x)[t]$. Since $K[x]$ is a UFD, hence by Gauss's lemma, $q(t)$ divides $R(x,t)$ in $K[x,t]$. Therefore $\deg_x(R)\geq\deg_x(q)$. 

On the other side, $$\deg_x(R)\leq\operatorname{ht}(u)=\max(\deg a_k,\deg b_k)\leq\max(\deg c_k,\deg c_n).$$ Therefore $\deg_x(R)\leq\deg_x(q)$. Hence, $\deg_x(R)=\deg_x(q)$, that is we have $$R(x,t)=q(t)m(t),$$ where $m\in K[t]$.

Then we will show that $m(t)$ is constant, i.e. $m(t)\in K$. Assume on the contrary $\deg(m)>0$, then we have $$a_k(t)=q_1(t)m(t)+l_1(t),b_k(t)=q_2(t)m(t)+l_2(t)$$ with $\deg(l_i)<\deg(m)(i=1,2)$. Since $m(t)$ divides $a_k(t)b_k(x)-a_k(x)b_k(t)$, hence also divides  $l_1(t)b_k(x)+a_k(x)l_2(t)$, which is possible only if $$l_1(t)b_k(x)+a_k(x)l_2(t)=0.$$ However, last equation is impossible since $a_k$ is coprime to $b_k$, and neither of them is constant. A contradiction! Hence, $\deg(m)=0$ and $m\in K$.

Write $q(t)=c\left(a_k(t)b_k(x)-a_k(x)b_k(t)\right),c\in K$, $[K(x):E]=\deg_t(q)=\operatorname{ht}(u)$. From lemma \ref{lem:lurothfinitedim}, we have $[K(x):K(u)]=\operatorname{ht}(u)$, hence we have $[K(x):K(u)]=[K(x):E]$. Since $u\in E$, then $E=K(u)=K(r_k)$.
\end{proof}

\section{Explicit Form of Invariants}
From now on, we have prove the existence of the $u(x)\in K(x)$, such that $E=K(u)$. Next, we will show the explicit form of $u$. Assume that $\operatorname{Gal}(K(x)/E)=\operatorname{Gal}(K(x)/(K(u)))=\{\sigma_1,\sigma_2,\dots,\sigma_n\}$. 

In fact, we have an easy proposition that if $F\subset E\subset K$ are fields, then $\operatorname{Aut}(K/E)$ is a subgroup of $\operatorname{Aut}(K/F)$, since the operation (composition) is the same
and any automorphism which fixes $E$ must also fix the smaller field $F$. So, we have 
$$\operatorname{Gal}(K(x)/E)=\operatorname{Aut}(K(x)/E)\subset\operatorname{Aut}(K(x)/K).$$

Next we will show that the automorphism group of a rational function field $K(x)$ is $\operatorname{PGL}(2,K)$. 
From lemma \ref{lem:lurothfinitedim}, let $u=\frac{ax+b}{cx+d}\in K(x)$, where $a,b,c,d\in K$ and $ad-bc\neq0$, then $[K(x)/K(u)]=\operatorname{ht}(u)=1$, i.e. $K(x)=K(u)$. On the other side, if we have $K(x)=K(u)$, i.e. $[K(x):K(u)]=1$, then $\operatorname{ht}(u)=1$. $u=\frac{P(x)}{Q(x)}$, where $\max{\deg(P(x)),\deg(Q(x))}=1$, then $u=\frac{ax+b}{cx+d}$, where $a,b,c,d\in K$ and $ad-bc\neq0$. i.e. We have $K(x)=K(u)$ if and only if $u\in \frac{ax+b}{cx+d}$, where $a,b,c,d\in K$ and $ad-bc\neq0$. 
i.e. we have the automorphism group of $K(x)$ is $$\operatorname{Aut}(K(x)/K)=\left\{x\mapsto\frac{ax+b}{cx+d}\mid a,b,c,d\in K, ad\neq bc\right\}.$$

Since general linear group is $$\operatorname{GL}(2,K):=\left\{\begin{pmatrix}a&b\\c&d\end{pmatrix}\mid a,b,c,d\in K, ad-bc\neq0\right\},$$ then consider the group homomorphism: $\operatorname{GL}(2,K)\rightarrow \operatorname{Aut}(K(x)/K)$, where $$\psi: \begin{pmatrix}a&b\\c&d\end{pmatrix}\mapsto \frac{ax+b}{cx+d},$$
with kernel $$\operatorname{Ker}\psi=\left\{\begin{pmatrix}a&0\\0&a\end{pmatrix}\mid a\in K-\{0\}\right\}.$$

We defined the projective general linear group $\operatorname{PGL}(2,K)=\operatorname{GL}(2,K)/(aI_2)$. Hence, we have $$\operatorname{Aut}(K(x)/K)\cong\operatorname{PGL}(2,K).$$ Now we have $\operatorname{Gal}(K(x)/E)\subset \operatorname{PGL}(2,K)$ up to isomorphism. 

We can easily write down the minimal polynomial of $x$ over $E=K(u)$ as following, 
$$p(t)=(t-\sigma_1(x))(t-\sigma_2(x))\cdots(t-\sigma_n(x))$$
Then $n=|\operatorname{Gal}(K(x)/E)|=[K(x):E]=[K(x):K(u)]=\operatorname{ht}(u)$.  

To sum up, we have following theorem:
\begin{theorem}
Suppose $K(x)/E$ is Galois, then $G=\operatorname{Gal}(K(x)/E)\subset\operatorname{PGL}(2,K)$. What's more, suppose the minimal polynomial of $K(x)/E$ is $$p(t)=(t-\sigma_1(x))(t-\sigma_2(x))\cdots(t-\sigma_n(x))=t^n+c_{n-1}t^{n-1}+\cdots c_1t+c_0,$$ where $c_i(x)\in K(x)$. Then $\forall c_i(x)\not\in K$ can be defined as $u=u(x)$, which satisfies $F(x)^G=F(u)$.
\begin{proof}
The first statement has been solved above, and the second statement is also a direct corollary of theorem \ref{thm:luroth-zwc}'s proof, since now the elementary symmetry polynomials of $\sigma_i(x)$ are the coefficients of the minimal polynomial of $x$ over $E=K(u)$.  At last, as for $F(x)^G=F(u)$, since $G$ is  a finite subgroup of $\operatorname{Aut}(K(x))$, then $K(x)/K(x)^G$ is Galois with Galois group $Gal(K(x)/(K(x)^G))=G$, hence $F(x)^G=F(u)$ which is directly from Artin's Theorem.
\end{proof}
\end{theorem}

We would like to give some examples. First of all, let us consider a simple case, that $\operatorname{Gal}(K(x)/E)=G=\{x\mapsto x,x\mapsto 1-x\}:=\{1,\sigma\}$, then the minimal polynomial of $x$ will be $p(t)=(t-x)(t-1+x)=t^2-t+x(1-x)$, hence $E=K(u)=K(x(1-x))$. Similarly, we can also find that there exists an isomorphism group of $G$, say $G'=\{x\mapsto x,x\mapsto 1/x\}=:\{1,\tau\}$, then $K(x)^{G'}=K(x+1/x)$.

Let us combine these two groups, then we will get a new group from the composition (in fractional linear transformation) of the elements in $G$ and $G'$, i.e. $$\hat{G}:=\langle\sigma,\tau\rangle=\{1,\sigma,\tau,\sigma\tau,\tau\sigma,\tau\sigma\tau=\sigma\tau\sigma\}\cong S_3\cong D_3.$$ Note that $\sigma,\tau,\sigma\tau\sigma$ are 2-cycles and $\sigma\tau,\tau\sigma$ are 3-cycles. In explicit, $$\hat{G}=\{x\mapsto x,x\mapsto 1/x,x\mapsto 1-x,x\mapsto 1/(1-x),x\mapsto 1-1/x,x\mapsto x/(x-1)\}.$$
Clearly, $\hat{G}$ is a finite subgroup of $\operatorname{PGL}(2,K)$, let $F(v)=F(x)^{\hat{G}}$, then consider the minimal polynomial of $x$ over $F(v)$:
$$p'(t)=(t-x)(t-(1-x))\left(t-\frac{1}{x}\right)\left(t-\frac{1}{1-x}\right)\left(t-\left(1-\frac{1}{x}\right)\right)\left(t-\frac{x}{x-1}\right)$$
Since the Norm and Trace of $x$ belong to $F$, we can find one of the nontrivial elementary polynomials of $g_i(x)\in\hat{G}$, $$\prod_{1\leq i< j\leq 6}g_i(x)g_j(x)=\frac{(x^2-x-1)^3}{x^2(1-x)^2}.$$ One can find that $\operatorname{ht}(v)=6=|\hat{G}|=|D_3|$. This example has been directly showed in Prof. Yu's course for Advanced Galois Theory.

As for more possibilities of the finite subgroup $\operatorname{Gal}(K(x)/K(u))$ of $\operatorname{Aut}(K(x)/K)=\operatorname{PGL}(2,K)$ is known. If $K$ is characteristics 0, then Klein showed that $G$ is either cyclic, dihedral, $A_4$, $S_4$ or $A_5$. If $K$ has characteristic $p>0$, then Dickson showed that the only other possibilities for $G$ are $\operatorname{PGL}(2,p^n)$,$\operatorname{PSL}(2,p^n)$, and subgroups of the group of upper-triangular matrices in $\operatorname{PGL}(2,p^n)$. Further, for any $K$, one knows explicitly all subgroups of $\operatorname{Aut}(K(x)/K)$ isomorphic to any of the above groups. 

%So now we just need to find a rational function $u$ in $E$, such that $\operatorname{ht}(u)=n$, then from similar process, we finish.

%Let consider all the symmetric polynomials of $g_i(x)$,

  \chapter*{Conclusions}
\label{chap:conclusion}
After sorting out the preliminaries for cyclic sextic field, we succeed in finding the integral basis for cyclic sextic field and giving an algorithm for prime decomposition. We also proposed many examples for cyclic cubic and sextic fields as well as some small theorems. On the other hand, using the Galois theory, we have prove two theorems for the Galois extensions over one-variable function field. The existence theorem is also a direct corollary of L\"{u}roth's theorem, and the explicit form given by the second theorem is very useful in finding the automorphism subgroups of a function field. 


%%%%%%%%%%%%%%%%%%%%%%%%%%%%%%
%% 附件部分
%%%%%%%%%%%%%%%%%%%%%%%%%%%%%%
\backmatter
  %结语
  
  % 参考文献
  % 使用 BibTeX
  % 选择参考文献的排版格式。注意ustcbib这个格式不保证完全符合要求,请自行决定是否使用
  \bibliographystyle{sustcbib}%{GBT7714-2005NLang-UTF8}
  \bibliography{bib/tex}
  \nocite{*} % for every item
  % 不使用 BibTeX
  % \include{chapter/bib}
  
  
  
  % 附录,没有请注释掉
  \begin{appendix}
  \chapter{Table of Orbit-length Partition for Small Degree Polynomials}
\label{chap:appD}
This table can be found in L. Soicher and J. McKay's \citep{soicher1985computing} articles.

\begin{table}[htbp]
\centering
\begin{tabular}{cccc}
\hline
$G$ & $x_1+x_2$ & $x_1+x_2+x_3$ & $x_1-x_2$\\
\hline
\textbf{Degree 3} &  &  &  \\
+$A_3$ &  & & $3^2$\\
$S_3$ &  & & 6\\

\textbf{Degree 4} &  &  &  \\
$Z_4$ & 2,4  & & $4^3$\\
+$V_4$ & $2^3$  & & $4^3$\\
$D_4$ & 2,4  & & 4,8\\
+$A_4$ &6  & & 12\\
$S_4$ &6  & & 12\\

\textbf{Degree 6} &  &  &  \\
$Z_6$ & 3,$6^2$  &2,$6^3$ & $6^5$\\
$S_3$ & $3^3$,6  &2,$6^3$ & $6^5$\\
$D_6$ & 3,$6^2$  &2,$6,12$ & 6,$12^2$\\
+$A_4$ &3,12  &$4^2,6^2$ & 6,$12^2$\\
$G_{18}$ &6,9  &2,18 & $6^2$,18\\
$G_{24}$ & 3,12 &$6^2$,8 &6,$12^2$\\
+$S_4/V_4$ & 3,12&$4^2$,12&6,24\\
$S_4/Z_4$ & 3,12 &8,12&6,24\\
$G_{36}^1$ & 6,9 &2,18&12,18\\
+$G_{36}^2$ & 6,9 &2,18&12,18\\
$G_{48}$& 3,12 &8,12 &6,24\\
+$\text{PSL}_2(5)$ & 15 & $10^2$ & 30\\
$G_{72}$ & 6,9 &2,18&12,18\\
$\text{PGL}_2(5)$ & 15&20&30\\
+$A_6$ & 15&20&30\\
$S_6$ & 15&20&30\\
\hline
\end{tabular}
\caption{Orbit-length Partition of Polynomial with Degree 3,4,6}
\label{tab:allorbits}
\end{table}

where ``+" means that the discriminants of the corresponding polynomial is square. Note that we need a non-linear resolvent to determine $PGL_2(5)$ and $S_6$ etc.
  \chapter{Resultant and Discriminant of a Polynomial}
\label{chap:appB}
We begin with some useful definitions regarding polynomials. The reason why we refer resultant is that we can calculate the discriminant of the polynomial through this tool, especially when the degree of polynomial is large. First of all, we give a definition of resultant\citep{healy2002resultants}:
\begin{definition}
Let $R$ be an integral domain, given two polynomials $f(x),g(x)\in R[x]$ with roots $\alpha_1,\dots,\alpha_m$ and $\beta_1,\dots,\beta_n$ respectively, then the resultant $\operatorname{Res}(f,g)$ of $f,g$ is defined to be $$\operatorname{Res}(f,g)=l(f)^nl(g)^m\prod_{i,j}(\alpha_i-\beta_j)$$ which is equivalent to both $$\operatorname{Res}(f,g)=l(f)^n\prod_{i}^m g(\alpha_i)$$ and $$\operatorname{Res}(f,g)=(-1)^{n m}l(g)^m\prod_{i}^n f(\beta_i)$$
\end{definition}
From this definition, on can easily see that $f,g$ have a common root in some if and only if $\operatorname{Res}(f,g)=0$.
An important proposition shows the relationship of this definition to the \textbf{Sylvester's matrix} (Some books take that as definition): Let $S$ be the Sylvester's matrix of polynomials $f(x)$ and $g(x)$, then $\operatorname{Res}(f,g)=\det(S)$.

Also for convenience, I choose a definition of normalized discriminant \citep{janson2007resultant} of polynomial as follows in this paper.

\begin{definition}
Let $f$ be a polynomial of degree $n\geq1$ with coefficients in a field $F$. Let $F_1$ be an extension of $F$ where $f$ splits, and let $r_1,\dots,r_n$ be the roots of $f$ in $F_1$. Then the discriminant of $f$ is $$\operatorname{Disc}(f):=\prod_{1\leq i<j\leq n}(r_i-r_j)^2$$
\end{definition}

Note further that $\operatorname{Disc}(cf)=\operatorname{Disc}(f)$ for any nonzero constant. The following theorem give the relation between the discriminant and resultant.

\begin{proposition}\label{prop:disctri}
Let $f=a_nx^n+\cdots+a_0$ be polynomial of degree $n\geq1$ with coefficients in a field $F$. The the discriminant of $f$ is given by 
$$\operatorname{Disc}(f)=(-1)^{n(n-1)/2}a^{-(2n-1)}_n \operatorname{Res}(f,f')$$
\end{proposition}

\begin{example}\label{for:nxpxq}
Let $f(x)=x^n+px+q$ for $n\geq2$, then $f'(x)=nx^{n-1}+p$, then from the Proposition \ref{prop:disctri}, we have $$\operatorname{Disc}(f)=(-1)^{(n-1)(n-2)/2}(n-1)^{n-1p^n}+(-1)^{n(n-1/2)}n^nq^{n-1}$$
\end{example}


  \chapter{The Legendre-Jacobi-Kronecker Symbol}
\label{chap:appA}
All the symbols with the form $\left(\frac{a}{b}\right)$ I used in the paper are Kronecker Symbols. But first, let us review the definition and some properties of Legendre symbol.
\section{The Legendre Symbol}
In number theory, the Legendre symbol is a multiplicative function with values {1,-1,0} that is a quadratic character modulo an \textbf{odd} prime number $p$: its value on a (nonzero) quadratic residue mod p is 1 and on a non-quadratic residue (non-residue) is −1. Its value on zero is 0. 
Furthermore, one can easily show that this symbol has the following properties:
\begin{proposition}
\begin{enumerate}
\item The Legendre symbol is periodic, if $a\equiv b(\text{mod }p)$, then $$\left(\frac{a}{p}\right)=\left(\frac{b}{p}\right)$$
\item The Legendre symbol is multiplicative, i.e. $$\left(\frac{a}{p}\right)\left(\frac{b}{p}\right)=\left(\frac{a b}{p}\right)$$
\item We have the congruence $a^{(p-1)/2}\equiv \left(\frac{a}{p}\right)(\text{mod }p).$
\item There are as many quadratic residues as non-residues mod $p$, say $(p-1)/2$.
\item Let $p$ be an odd prime, then $$\left(\frac{-1}{p}\right)=(-1)^{(p-1)/2}, \left(\frac{2}{p}\right)=(-1)^{(p^2-1)/8}$$
\item Let $p,q$ be two different odd primes, then we have reciprocity law: $$\left(\frac{p}{q}\right)\left(\frac{q}{p}\right)=(-1)^{(p-1)(q-1)/4}$$
\end{enumerate}
\end{proposition}

\section{The Kronecker Symbol}
Now we extend the definition of the Legendre symbol \citep{cohen1993course}.
\begin{definition}
we define the Kronecker (or Kronecker-Jacobi) symbol $\left(\frac{a}{b}\right)$ for any $a$ and $b$ in $\mathbb{Z}$ as follows:
\begin{enumerate}
\item If $b=0$, then $\left(\frac{a}{0}\right)=1$ if $a=\pm1$, and is equal to 0 otherwise.
\item For $b\neq0$, firstly $\left(\frac{a}{1}\right)=1$. For other case write $b=\prod p$, where $p$ are not necessarily distinct primes (including $p=2$), or $p=-1$ to take care of sign. The we set $$\left(\frac{a}{b}\right)=\prod\left(\frac{a}{p}\right),$$ where $\left(\frac{a}{p}\right)$ is the Legendre symbol defined above for $p>2$, and where $p=2$ we define: $$\left(\frac{a}{2}\right)=\left\{\begin{array}{cc} 0, & \text{if } a \text{ is even}\\
(-1)^{(a^2-1)/8}, &  \text{if } a \text{ is odd.}
\end{array}\right.$$
and also $$\left(\frac{a}{-1}\right)=\left\{\begin{array}{cc} 1, & \text{if } a\geq0\\
-1, &  \text{if } a<0.
\end{array}\right.$$
\end{enumerate}
\end{definition}

Also the Kronecker symbol has the following simple properties:
\begin{proposition}
\begin{enumerate}
\item $\left(\frac{a}{b}\right)=0$ iff $(a,b)\neq1$
\item for all $a,b,c$, if $b c\neq0$, we have $$\left(\frac{ab}{c}\right)=\left(\frac{a}{c}\right)\left(\frac{b}{c}\right), \left(\frac{a}{bc}\right)=\left(\frac{a}{b}\right)\left(\frac{a}{c}\right)$$
\end{enumerate}
\end{proposition}
  
  
  \end{appendix}
   
  \begin{thanks}
This thesis is the result of many years of study whereby I have been accompanied and supported by many people. It is a pleasure to convey my gratitude to all of them in my humble acknowledgement.

I want to give my sincerely thanks to my graduation designation supervisor Prof. Jingzhi Li for his earnest guidance and great help during my undergraduate study at SUSTC.

I would especially like to recognize my academic supervisors, Prof. Xianke Zhang and Prof. Jietai Yu, who have always given me encouragements and confidence in myself. They have played important roles in my decision to continue my education and inspired me for what I could achieve in the future. During the preparation of the thesis, we had many fruitful discussion. They have given me so many advises on the details of the thesis, including the tentative methods and writing techniques etc.

I am very grateful for all the encouragement I have received from my family and from a network of friends.  I would also like to thank my wonderful committee of professors. The committee members Prof. Xianke Zhang, Prof. Jingzhi Li, Prof. Anyue Chen, Prof. Xuejun Jiang and Prof. Linlin Su, who have contributed to the completion of this thesis through their guidance and exceptional teaching, which have helped me get through my education this far.  


Finally, I'd like to say thanks to all my teachers in SUSTC, especially teachers in the Department of Financial Mathematics for providing a stimulating environment. I am very grateful for all the encouragement I have received from my family and from a network of friends. 



\vskip 18pt

\begin{flushright}

Wenchao Zhang

September,2014

\end{flushright}

\end{thanks}


  

\end{document}
