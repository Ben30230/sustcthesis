\chapter{Quadratic Fields}
\label{chap:chap-four}
In this chapter, we are going to discover the most simplest number field that are different from
$\mathbb{Q}$, i.e. quadratic fields. Let denote it as $K$. A quadratic field is of degree 2 over $\mathbb{Q}$, it can be given by $K=\mathbb{\theta}$, where $\theta$ is a root of minimal polynomial $f(x)=x^2+a x+b$ of $\mathbb{Z}[x]$. Let $d=a^2-b$, then $K=\mathbb{Q}(\sqrt{d})$. Clearly, $d$ is not a square, otherwise $f(x)$ won't be a irreducible quadratic polynomial. Since $\mathbb{Q}(\sqrt{m^2d})=\mathbb{Q}(\sqrt{d})$, hence we may assume that $d$ is squarefree.

Since $n=2=r_1+2r_2$, the signature of quadratic field $K$ is either $(2,0)$ or $(0,1)$, which is called real quadratic field or complex (or imaginary) quadratic field respectively.
It's easy to show that the real quadratic field has positive discriminant and the complex quadratic field has negative discriminant.

\section{Discriminant, Integral Basis}
The integral basis and discriminant of quadratic field is easy as following theorem:
\begin{theorem}\label{thm:quad_disc}
Let $K=\mathbb{Q}(\sqrt{d})$ be a quadratic field with $d$ squarefree. Then if $d\equiv1(\operatorname{mod} 4)$, we have integral basis $(1,(1+\sqrt{d})/2)$ with discriminant $d(K)=d$; if $d\equiv2\text{ or }3(\operatorname{mod} 4)$, we have integral basis $(1,\sqrt{d})$ with discriminant $d(K)=4d$.
\end{theorem}

Let us denote $D=d(K)$ (namely fundamental discriminant) for these two cases, then we can see that it satisfies Stickelberger's criterion \ref{thm:stickelberger}. What's more, $K=\mathbb{Q} (\sqrt{D})$ has integral basis $(1,\omega)$, where $$\omega=\frac{D+\sqrt{D}}{2},$$ and therefore $O_K=\mathbb{Z}[\omega]$.

\section{Decomposition of Primes}
Note that Proposition \ref{thm:decomposition} and Proposition \ref{thm:kummer} immediately shows how prime numbers decompose in a quadratic field \citep{Xianke2006ANT}.


\begin{theorem}\label{thm:quad_decompose}
Let $K=\mathbb{Q}(\sqrt{D})$, where $D$ is the fundamental discriminant, i.e. $D=d(K)$, $O_K=\mathbb{Z}[\omega]$ where $\omega=(D+\sqrt{D})/2$ its ring of integers, and $p$ be a prime number. Then\footnote{$\left(\frac{D}{p}\right)$ is Kronecker symbol which can be seen in Appendix \ref{chap:appA}.}
\begin{enumerate}
\item If $\left(\frac{D}{p}\right)=0$, then $p$ is ramified, i.e. $pO_K=\wp^2$. More precisely, $$\wp=pO_K+\omega O_K,$$ except when $p=2$ and $D\equiv12(\operatorname{mod} 16)$.
\item If $\left(\frac{D}{p}\right)=-1$, then $p$ is inert, hence $pO_K=\wp$ is a prime ideal.
\item If $\left(\frac{D}{p}\right)=1$, then $p$ is split, and we have $pO_K=\wp_1\wp_2$, where 
$$\wp_i=pO_K+(\omega-\frac{D\pm b}{2})O_K,$$ and $b$ is any solution to $b^2\equiv D(\operatorname{mod} 4p)$
\end{enumerate}
\end{theorem}

\section{Unit Group of Imaginary Quadratic Fields}
First we consider the unit group of quadratic field, by the Dirichlet's Unit theorem \ref{thm:Dirichlet}, we can divide the quadratic field into two cases. The imaginary quadratic field has simple unit group since $r=r_1+r_2-1=0$, i.e. the unit group is finite. In fact, the imaginary quadratic field is the only number field apart from $\mathbb{Q}$ who has finite units. A theorem for the unit group of Imaginary Quadratic fields could be found in \citep{Xianke2006ANT,cohen1993course} etc.:

\begin{theorem}\label{thm:quad_unit}
Let $K=\sqrt{D}$, where the fundamental discriminant $D<0$, then the group $U(K)=\mu(K)$ of units  is equal to the group of $\omega(D)-th$ roots of unity, where 
$$\omega(D)=\left\{\begin{array}{ll}
2,& \text{if }D<-4\\
4,& \text{if }D=-4\\
6,& \text{if }D=-3
\end{array}\right.$$
\end{theorem}

\section{Class number of Imaginary Quadratic Fields}
Let us now consider the problem of computing the class number of imaginary quadratic field. Here we give a beautiful result from L-function. Since it's to far to enter into the details of the analytic theory of \textbf{L-functions}, so we just recall the results. Our main result is a corollary of Dirichlet's Theorem (We have reorganized this theorem to fit all imaginary quadratic number field. In the original theorem, it just states $D<-4$ cases.) \citep{cohen1993course}.
 
\begin{theorem}
If $D$ is a negative fundamental discriminant, then $$h(K)=\frac{\omega(D)}{4-2\left(\frac{D}{2}\right)}\sum_{1\leq r<|D|/2}\left(\frac{D}{r}\right),$$ where $\omega(D)$ is defined in Theorem \ref{thm:quad_unit}. 
\end{theorem} 

\begin{remark}
For $D<-4$, we have $$h(K)=\frac{1}{2-\left(\frac{D}{2}\right)}\sum_{1\leq r<|D|/2}\left(\frac{D}{r}\right).$$ Of course, we can get $h(-3)=h(-4)=1$ by calculation. 
\end{remark}
More precise list for class number can be seen in the book \citep{Xianke2006ANT}.

\section{Unit group of Real Quadratic Fields}
For real quadratic fields, ZHANG \citep{Xianke2006ANT} has given a precise algorithm to compute it through the Pell's equation. Since the unit circle of a real quadratic field only hits two time real axis, then we have $\mu{K}=\{\pm1\}$. By Dirichlet Unit Theorem \ref{thm:Dirichlet}, we have $$U(K)=\{\pm1\}\times \epsilon^{\mathbb{Z}},$$ where $\epsilon$ generates an infinite cyclic group $\langle\epsilon\rangle=\epsilon^{\mathbb{Z}}$. There exists only one unit in the set (all of them can generate whole infinite cyclic group) $\{\pm\epsilon,\pm\epsilon^{-1}\}$ which is larger than 1, and we denote it as fundamental unit of $K$. 
Now we just need to compute the fundamental unit. Let $K=\mathbb{Q}(\sqrt{d})$, where $d$ is squarefree. Let $\alpha=a+b\sqrt{d}(a,b\in\mathbb{Q})$ be unit of $K$, then $N(\alpha)=\pm1$. 
If $d\equiv 2 \text{ or }3 \mod 4$, then $O_K=\mathbb{Z}[\sqrt{d}]$. A integer $\alpha=x+y\sqrt{d}(x,y\in\mathbb{Z})$ is unit if and only if $N(\alpha)=\pm1$, i.e. we have following Pell's equation:
\begin{equation}\label{eqn:pell1}
x^2-d y^2=\pm1
\end{equation} 
From the Unit Theorem, we note that if $\epsilon=a_1+b_1\sqrt{d}$ is a fundamental unit of $K$($a_1,b_1>0$), then $$\epsilon^n=(a_1+b_1\sqrt{d})^n=a_n+b_n\sqrt{d}$$ is also a unit who larger than one. What's more, $(a_n,b_n)$ are natural number solutions of the Pell's equation \ref{eqn:pell1}. Note that $b_{n+1}=a_1b_n+a_nb_1$, hence $b_n$ is an increasing sequence.

Let us consider $b=1,2,3,\cdots$, if $d b^2\pm1$ is a square number (i.e. $a^2$), then we stop the process, i.e. we find the "smallest" solution of the Pell's equation \ref{eqn:pell1} and this solution is also the fundamental unit.

Let's consider $d=6$, $6b^2\pm1$ is a square number firstly when $b=2$, i.e. we have the fundamental unit of $\mathbb{Q}(\sqrt{6})$ is $\epsilon=5+2\sqrt{6}$.

Similarly, if $d\equiv1 \mod 4$, then $O_K=\mathbb{Z}[\frac{1+\sqrt{d}}{2}]$, an integer of $K$ has form $\alpha=(a+b\sqrt{d})/2$, where $a\equiv b \mod 2$, $a,b\in\mathbb{Z}$. $\alpha$ be unit of $K$ if and only if $N(\alpha)=(a^2-b^2d)/4=\pm1$, hence, $(a,b)$ is a solution of the following Pell's equation:
\begin{equation}\label{eqn:pell2}
x^2-d y^2=\pm4
\end{equation}
if $\epsilon=(a_1+b_1\sqrt{d})/2$ is a fundamental unit of $K$($a_1,b_1>0$), then $$\epsilon^n=(\frac{a_1+b_1\sqrt{d}}{2})^n=\frac{a_n+b_n\sqrt{d}}{2}$$ is also a unit who larger than one. Similarly, consider $b=1,2,3,\cdots$, if $d b^2\pm4$ is a square number (i.e. $a^2$), then we stop the process and get the fundamental unit $(a+b\sqrt{d})/2$.

\section{Class Numbers of Real Quadratic Fields}
Also, as in the imaginary case, using L-function, we can also get a beautiful results\citep{cohen1993course}. We have modified the results through discarding regulators.

\begin{theorem}\label{thm:rquad_cl}
If $D$ is a positive fundamental discriminant, then $$h(K)=-\frac{1}{\operatorname{ln}(\epsilon)}\sum_{r=1}^{\lfloor(D-1)/2\rfloor}\left(\frac{D}{r}\right)\operatorname{ln}\sin\left(\frac{r\pi}{D}\right).$$
\end{theorem}

Now we see an example, let $D=4d=8$, then $\epsilon=1+\sqrt{2}$ by the result of unit group of real quadratic field. From the Theorem \ref{thm:rquad_cl}, we have $$h(K)=-\frac{1}{\operatorname{ln}(1+\sqrt{2})}\left(\operatorname{ln}\sin(\frac{\pi}{8})-\operatorname{ln}\sin(\frac{3\pi}{8})\right)=-\frac{\operatorname{ln}(\tan(\frac{\pi}{8}))}{\operatorname{ln}(1+\sqrt{2})}=1.$$

